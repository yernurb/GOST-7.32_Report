\Introduction

В природе гранулярная материя является одним из самых распространенных типов вещества, начиная от песка под нашими ногами,
сахара для чая, различных порошков для строительства и техногенного производства, заканчивая космической пылью в аккреционных дисках 
зарождающихся звездных и галактических системах. Гранулярная материя характеризуется в основном диссипативными свойствами при контактном
взаимодействии составных частиц. Частный случай гранулярных систем, так называемый \emph{гранулярный газ}, является объектом интереса
в нашей работе \cite{Brilliantov:2004book}. Под газообразной мы будем подразумевать систему в которой все контактные взаимодействия бинарные, 
т.е. в любой момент времени во всех взаимодействиях участвуют только два объекта, а тройные, четверные и т.д. взаимодействия исключены.
Таким образом, подобная система может быть описана классическими уравнениями Больцмана-Энскога.

Объектом наших исследований являются кольца Сатурна, которые представляют собой один из ярких примеров гранулярных газов в природе. 
Сами кольца состоят в основном из водяного льда и силикатных образований. Размеры частиц материала кольца составляют от микрометров до нескольких 
десятков метров. Объекты б\'{о}льших размеров, от нескольких сот метров до километров и более, классифицируются уже как отдельные луны Сатурна. 
Некоторые из подобных лун, как Пан и Дафнис, 
