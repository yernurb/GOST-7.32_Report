\Conclusion % заключение к отчёту
Мы рассмотрели полидисперсную систему гранулярных газов во вращающемся планетарном диске. Основной фокус работы был направлен на 
исследование влияния полидисперсности системы на возникающую динамику. Главным эффектом полидисперсности является нарушение 
равнораспределения энергии системы, что в свою очередь приводит к неодинаковости гранулярных температур подсистем. 
Мы написали кинетические уравнения, описывающие эволюцию функции распределения каждой из подсистем, и с помощью них получили
гидродинамические уравнения в линейном приближении по малому пространственному параметру, или в т.н. приближении Навье-Стокса.
Также мы обсудили существование нормального решения системы, в котором после некоторого релаксационного периода, каждая
из подсистем эволюционирует в независимости от остальной системы, и их скорости охлаждения оказываются одинаковыми. Однако 
во время периода релаксации происходит нарушение равнораспределения энергии, и подсистемы находятся в разных энергетических состояниях,
что в конечном итоге приводит все подсистемы к различным стационарным состояниям. 

Также, мы рассмотрели нашу систему в в центральном гравитационном поле планеты, т.е. в системе планетарных колец, и ограничились 
рассмотрением системы как двумерного плоского диска с дифференциальным вращением. В этом случае, диссипация системы компенсируется
энергией гравитационного сдвига, когда часть системы с более высокой энергией орбитального движения, передает энергию другой части
системы с меньшей энергией. Данная передача энергии происходит в виде вязкого трения между соседними слоями диска. В итоге, 
данная подкачка энергии компенсирует диссипативные потери энергии, и все подсистемы приходят к различным стационарным состояниям.

Также, мы построили компьютерную модель, для симуляции данной системы, с помощью метода дискретных элементов. В этой модели,
динамика каждой частицы, или элемента, рассчитывалась отдельно, и для каждого из этих элементов решались уравнения движения.
Однако на каждом шаге эволюции, необходимо было следить за всеми возможными взаимодействиями со стороны других элементов системы.
При возникновения контакта между элементами, рассчитывалась результирующая сила взаимодействия, в нашем случае это были
сила упругости Герца, и сила вязкого трения приводящая к диссипации. Также, необходимо было учитывать гравитационное воздействие
центрального тела на каждый элемент. В итоге нам пришлось решать уравнения Хилла, для симуляции нашей системы. Результаты 
симуляции подтвердили наши теоретические предположения об установлении различных температур в каждой из подсистем, 
и все приведенные аппроксимации и предположения о системе можно считать верными.

%%% Local Variables: 
%%% mode: latex
%%% TeX-master: "rpz"
%%% End: 
