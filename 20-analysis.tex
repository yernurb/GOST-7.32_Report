\chapter{Кинетическое описание полидисперсной системы}
\label{cha:analysis}

Для статистического описания неравновесной системы, мы будем исходить из кинетических уравнений Больцмана.
Перед этим нам необходимо определить фазовое пространство в котором происходит эволюция динамики отдельно
взятой частицы, а также более детально рассмотреть механику столкновений для составления интегралов столкновений.

\section{Уравнения Больцмана}

Рассмотрим весь газ как смесь из однородных газов, массы частиц в которых обозначим через $m$. Пусть $N_m$ будет 
количеством частиц массы $m$ в системе. Если общее количество частиц равно $N$, то
\begin{equation}
  \eta_m = \frac{N_m}{N}\;,
\end{equation}
относительное количество частиц массы $m$ в полной системе. Учитывая что количество частиц в планетарных кольцах огромно,
так же как и разновидность частиц по массе, то мы можем ввести функцию распределения особей в системе по массам как
\begin{equation}
  \eta(m) = \lim_{N\to\infty}\frac{N(m)}{N}\;,
\end{equation}
либо
\begin{equation}\label{eq:mass_distribution}
  \int\eta(m)\,dm=1\;.
\end{equation}
Здесь следует отметить, что как видно из \ref{eq:mass_distribution}, $\eta(m)\to 0$ при $m\to\infty$, что согласовывается 
с реальными наблюдениями. Таким образом, масса частицы $m$ будет показывать род подсистемы как непрерывная переменная.

Динамика отдельно взятой частицы массы $m$ описывается его векторами координат $\br$ и скоростей $\bv$ в фазовом пространстве.
В этом фазовом пространстве введем функцию распределения $f(t,m,\br,\bv)$ которое имеет следующее важное свойство:
\begin{equation}
  dN(t, m,\br,\bv) = f(t, m, \br, \bv)d\bv\;,
\end{equation}
где $dN(t, m, \br, \bv)$ -- функция числа частиц локализованных вокруг координаты $\br$ и имеющих скорости в диапазоне от
$\bv$ до $\bv+d\bv$. 
Так как газообразные системы являются разряженными, то макропараметры системы могут быть определены как
некие интегралы от одночастичной функции распределения. 

Нулевой момент дает нам функцию количественной плотности частиц
\begin{equation}
  n(t, m,\br) = \int f(t, m,\br,\bv)\,d\bv\;,
\end{equation}
либо функцию плотности масс
\begin{equation}
  \rho(t, m,\br) = mn(t, m,\br)=\int mf(t, m,\br,\bv)\,d\bv\;,
\end{equation}
и так далее. Более подробно макропараметры системы будут описаны при гидродинамическом описании системы.
Здесь же мы видим что параметры системы нестационарны и зависят от времени $t$, тогда как сама эволюция
функции распределения по времени подчиняется уравнению Больцмана \cite{Spahn:2004euro_lett_kinetic_fraggr,Dilley:1993icarus_energy_loss}:
\begin{equation}\label{eq:Boltzmann_general}
  \frac{\pd f}{\pd t} + \bv\frac{\pd f}{\pd\br} + \bw\frac{\pd f}{\pd v} = I_c(t,m,\br,\bv)\;,
\end{equation}
где $I_c(t,m,\br,\bv)$ -- полный интеграл столкновений, $\bw$ -- ускорение частицы под воздействием внешних сил.
Полный интеграл столкновений пишется через бинарный интеграл столкновений как:
\begin{equation}
  I_c(t,m,\br,\bv) = \int\eta(m')I_c(t,m',m,\br,\bv)\,dm'\;.
\end{equation}
Для нахождения точной формы интегралов столкновений необходимо более подробно изучить механику самих столкновений частиц.

\section{Механика столкновений}

Везде в дальнейшем мы будем предполагать, что все частицы в газе являются абсолютно сферическими и однородными,
с одинаковыми коэффициентами реституции $\eps$. При столкновении двух таких частиц, закон сохранения импульса запишется как:
\begin{equation}\label{eq:momentum_conservation}
  m_i\bv_i + m_j\bv_j = m_i\bv'_i+m_j\bv'_j\;,
\end{equation}
где знаками штриха $'$ обозначены скорости частиц после столкновения. Обмена массами и прилипания не происходит по условию задачи.
Для гранулярных газов закон сохранения энергии (механической) нарушается, а изменение относительных скоростей задается как:
\begin{equation}
  \bg'_{n} = -\eps\bg_{n}\;,
\end{equation}
где $\bg = \bg_{ij}=\bv_i-\bv_j$, а $\bg_n$ -- является нормальной составляющей относительной скорости. Определим нормальное 
направление столкновения как проходящей через центры сталкивающихся частиц в момент столкновения, и введем вектор
нормали:
\begin{equation}
  \bn = \bn_{ij} = \frac{\br_i-\br_j}{\sigma_i-\sigma_j}\;,
\end{equation}
где $\br_i,\;\br_j$ -- координаты частиц в момент столкновения.
Далее запишем изменение относительных скоростей следующим образом:
\begin{equation}
  (\bg'\cdot\bn)\bn = -\eps(\bg\cdot\bn)\bn\;,
\end{equation}
и подставляя в (\ref{eq:momentum_conservation}) получаем скорости частиц после столкновения:
\begin{equation}
  \begin{split}
    \bv'_i &= \bv_i - \frac{\mu}{m_i}(1+\eps)(\bg\cdot\bn)\bn\;,\\
    \bv'_j &= \bv_j + \frac{\mu}{m_j}(1+\eps)(\bg\cdot\bn)\bn\;,
  \end{split}  
\end{equation}
где $\mu=\mu_{ij}=\cfrac{m_im_j}{m_i+m_j}$ -- эффективная масса столкновения. Здесь следует иметь ввиду что вектор $\bn$
свободен и не зависит от значений скоростей.

Рассмотрим теперь изменение импульса частицы после столкновения:
\begin{equation}
  \delta\bm{p}_i = -\delta\bm{p}_j = \pm\mu(1+\eps)(\bg\cdot\bn)\bn\;.
\end{equation}
Знак переноса импульса зависит от взаимной конфигурации векторов $\bg$ и $\bn$, однако полное изменение импульса
конечно же $\delta\bm{p}_i+\delta\bm{p}_j=0$.

Теперь рассмотрим изменение кинетической энергии после столкновения:
\begin{equation}\label{eq:delta_E_v}
  \begin{split}
    \delta E_i &= \frac{m_i\bv^{'2}_i}{2}-\frac{m_i\bv^2_i}{2} 
    = -\mu(1+\eps)(\bg\cdot\bn)(\bv_i\cdot\bn)+\frac{\mu^2}{2m_i}(1+\eps)^2(\bg\cdot\bn)^2\;, \\
    \delta E_j &= \frac{m_j\bv^{'2}_j}{2}-\frac{m_j\bv^2_j}{2} 
    = +\mu(1+\eps)(\bg\cdot\bn)(\bv_j\cdot\bn)+\frac{\mu^2}{2m_j}(1+\eps)^2(\bg\cdot\bn)^2\;, \\
  \end{split}
\end{equation}
или перейдя в систему отсчета центра масс со скоростью $M\bv_C=m_i\bv_i+m_j\bv_j$, где $M=M_{ij}=m_i+m_j$, получим:
\begin{equation}
  \begin{split}
    \bv_i &= \bv_C + \frac{\mu}{m_i}\bg\;,\\
    \bv_j &= \bv_C - \frac{\mu}{m_j}\bg\;,
  \end{split}
\end{equation}
и соответственно
\begin{equation}\label{eq:delta_E}
  \begin{split}
    \delta E_i &= -\mu(1+\eps)(\bg\cdot\bn)(\bv_C\cdot\bn)-\frac{1-\eps^2}{2}\frac{\mu^2}{m_i}(\bg\cdot\bn)^2\;,\\
    \delta E_j &= +\mu(1+\eps)(\bg\cdot\bn)(\bv_C\cdot\bn)-\frac{1-\eps^2}{2}\frac{\mu^2}{m_j}(\bg\cdot\bn)^2\;.\\
  \end{split}
\end{equation}
Как видим, первые члены в правой части уравнений одинаковы и противоположны в знаках. Это означает что данная часть
кинетической энергии передается от одной частицы к другой, и остается в самой системе, не диссипируя. Вторые же
члены как мы видим всегда отрицательны и при разных массах $m_i\neq m_j$ также отличны друг от друга. Именно эта 
часть отвечает за диссипацию энергии в системе, и диссипация тем больше чем меньше коэффициент реституции $\eps$,
т.е. чем менее упругим будет столкновение, что вполне ожидаемо. Однако, как мы видим, диссипация также зависит
и от массы самой частицы, и чем \emph{меньше} масса частицы, тем \emph{больше} потери энергии:
\begin{equation}
  \left(\frac{\delta E_i}{\delta E_j}\right)_{diss}=\frac{m_j}{m_i}\;.
\end{equation}
Именно этот эффект приводит к нарушению равнораспределения энергии в полидисперсной системе гранулярных газов,
что в свою очередь приводит к неодинаковости гранулярных температур подсистем.
Полная потеря энергии при столкновении равна
\begin{equation}
  \delta E_i+\delta E_j = -\frac{1-\eps^2}{2}\mu(\bg\cdot\bn)^2\;.
\end{equation}

\section{Интеграл столкновений}
После описания механики столкновений, перейдем к самому виду интеграла столкновений в (\ref{eq:Boltzmann_general}).
В общем виде интеграл столкновений для гранулярных газов имеет вид:
\begin{equation}
  \begin{split}
    I_c(t,m_i,\br,\bv_i) &= \int dm_jg_2(\sigma_{ij})\eta(m_j)\sigma^{D-1}_{ij}\int d\bv_j\int d\bn\Theta(-\bg\cdot\bn)\vert\bg\cdot\bn\vert\times\\
    &\times\left(\frac{1}{\eps^2}f(t,m_i,\br,\bv''_i)f(t,m_j,\br,\bv''_j)-f(t,m_i,\br,\bv_i)f(t,m_j,\br,\bv_j)\right)\;,
  \end{split}
\end{equation}
где $\sigma_{ij}=\sigma_i+\sigma_j$ -- расстояние между центрами частиц, $g_2(\sigma_{ij})$ -- параметр Энскога, который учитывает
разницу в координатах центров частиц во время столкновений, который мы примем равным единице $g_2(\sigma_{ij})=1$, $\Theta(x)$ -- 
функция Хевисайда, которая включена для того чтобы учитывать только те соотношения скорости, при которых они сближаются, 
$\bv''_i,\;\bv''_j$ -- скорости обратных столкновений, $D$ -- размерность системы. По смыслу, интеграл столкновений показывает изменения 
в функции распределения за счет столкновений частиц. Интеграл столкновений имеет одно важное свойство, а именно, если взять некую 
динамическую функцию системы $\psi_i(\bv_i)$, и ее изменение после после прямого столкновения записать как 
$\Delta\psi_i(\bv_i)=\psi_i(\bv'_i)-\psi_i(\bv_i)$, то получим:
\begin{equation}\label{eq:collision_integral_dynamic_function}
  \begin{split}
    \frac{d}{dt}\langle\psi_i(\bv_i)\rangle_c &= \int\psi_i(\bv_i)\frac{\pd f_i}{\pd t}\,d\bv_i = \int\psi_i(\bv_i)I_c(t,m_i,\br,\bv_i)\,d\bv_i=\\
    &=\int dm_j\eta(m_j)\sigma^{D-1}_{ij}\int d\bv_id\bv_j\int d\bn\Theta(-\bg\cdot\bn)\vert\bg\cdot\bn\vert\times\\
    &\times f(t,m_i,\br,\bv_i)f(t,m_j,\br,\bv_j)\Delta\psi_i(\bv_i)\;.
  \end{split}
\end{equation}
Это свойство интеграла столкновения мы будем использовать в дальнейшем при описании эволюции энергии всей системы в целом.

\section{Эволюция энергии системы}
Для того чтобы оценить изменение энергии всей системы в целом за счет столкновений, возьмем функцию кинетической энергии
частицы как $\psi_i(\bv_i) = \cfrac{m_i\bv^2_i}{2}$, а вместо $\Delta\psi_i(\bv_i)=\delta E_i$ из уравнения (\ref{eq:delta_E}) и подставим
в (\ref{eq:collision_integral_dynamic_function}). В итоге получаем:
\begin{equation}\label{eq:energy_collision_evolution}
  \begin{split}
    \int\frac{m_iv^2_i}{2}I_c(t,m_i,\br,\bv_i)\,d\bv_i &= \int dm_j\eta(m_j)\sigma^{D-1}_{ij}
    \int d\bv_id\bg\int d\bn\Theta(-\bg\cdot\bn)\vert\bg\cdot\bn\vert\times\\
    &\times f(t,m_i,\br,\bv_i)f(t,m_j,\br,\bv_j)\delta E_i(\bv_i,\bg)\;,
  \end{split}
\end{equation}
здесь мы сделали замену переменных $d\bv_id\bv_j=d\bv_id\bg$. Для дальнейшего продвижения, нам необходимо знать вид самой функции распределения.

