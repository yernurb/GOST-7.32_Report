\chapter{Гидродинамическое описание системы}
\label{cha:design}

Перейдем теперь к более грубому, гидродинамическому описанию системы. Для начала нужно определить возможность такого перехода
в рассматриваемой нами системе. Через $L$ обозначим линейный размер всей системы в целом. Для кинетического описания нам было достаточно
что линейные размеры частиц $\sigma$ были намного меньше размера всей системы, т.е. выполнение условия $\sigma\ll L$. Соответственно
все кинетические уравнения пишутся на уровне детализации $\sigma$. Однако гидродинамическое описание происходит на другом уровне,
который мы обозначим $\ell$, и который удовлетворяет условию $\sigma\ll\ell\ll L$. Если мы можем вести изучение системы в таком масштабе,
то можно говорить что мы рассматриваем систему в гидродинамическом приближении. Линейные размеры колец Сатурна, т.е. ширина колец,
растягивается примерно на $L\sim 66 000$ км, в то время как размеры самих частиц варьируются в пределах $\sigma\sim 10^{-2}\div 10^2$ см.
Отсюда хорошо видно что можно подобрать такое значение $\ell\sim 1\div 2$ км, в пределах которого гидродинамическое описание системы
будет вполне оправдано. Таким образом, в дальнейшем мы можем говорить что в пределах $\ell$ вокруг координаты $\br$ находится достаточно большое
количество частиц, по которым можно определить макропараметры системы в зависимости от самой координаты $\br$. Перейдем к непосредственным 
определениям самих макропараметров, необходимых для полного гидродинамического описания системы, и написанию уравнений переноса
этих параметров.

\section{Уравнения переноса}

Зная функцию распределения, можно вводить пространственно распределенные параметры, описывающие систему в целом, а не через отдельно
взятые частицы. Такие параметры называются макропараметрами системы. Мы будем вводить их как моменты вектора скорости частиц $\bv$. Так,
нулевой момент, как мы уже ранее определили, дает нам функцию числовой плотности:
\begin{equation}
  n(t,m,\br) = \int f(t,m,\br,\bv)\,d\bv\;,
\end{equation}
либо функцию плотности масс
\begin{equation}\label{eq:mass_density}
  \rho(t,m,\br) = mn(t,m,\br) = \int mf(t,m,\br,\bv)\,d\bv\;.
\end{equation}
Данная функция читается так: $\rho(t,m,\br)$ -- это масса в единичном объеме участка кольца в координате $\br$ во время $t$, при этом 
масса вещества данного участка кольца равна $m$. Остальные параметры имеют схожий физический смысл.

Первый момент по скорости дает нам функцию плотности импульса
\begin{equation}\label{eq:momentum_density}
  \rho\bu(t,m,\br)=\int m\bv f(t,m,\br,\bv)\,d\bv\;,
\end{equation}
где $\bu$ -- дает нам среднюю скорость участка кольца. Введем понятие локальной скорости частиц
\begin{equation}\label{eq:local_velocity}
  \bc = \bv-\bu(t,m,\br)\;,
\end{equation}
которая показывает скорость (хаотического движения) частиц в системе отсчета движущейся вместе с участком кольца со скоростью $\bu$.
Таким образом введем понятие \emph{гранулярной температуры} системы, по аналогии с термодинамической температурой как второй момент
по скорости
\begin{equation}\label{eq:granular_temperature}
  \frac{D}{2}nT(t,m,\br) = \int\frac{m\bc^2}{2}f(t,m,\br,\bv)\,d\bv\;,
\end{equation}
где $D=3$ -- для трехмерной системы, однако мы будем рассматривать двумерную систему $D=2$.
Нам также понадобятся и другие моменты по скорости, которые уже являются тензорными величинами. Во первых, это
тензор \emph{напряжения}:
\begin{equation}
  \Pi_{\ab}(t,m,\br)=m\int v_\alpha v_\beta f(t,m,\br,\bv)\,d\bv\;,
\end{equation}
который можно разделить на две части используя (\ref{eq:local_velocity})
\begin{equation}
  \Pi_{\ab}(t,m,\br)=\rho u_\alpha u_\beta+P_{\ab}\;.
\end{equation}
Первая часть является динамической частью тензора напряжения, а вторая часть называется тензором \emph{внутренних напряжений}:
\begin{equation}
  P_{\ab}(t,m,\br)=m\int c_\alpha c_\beta f(t,m,\br,\bv)\,d\bv\;.
\end{equation}
Разделяя далее данный тензор на часть с нулевой сверткой и на диагональную часть, получаем:
\begin{equation}
  P_{\ab} = \delta_{\ab}p_{id}+\pi_{\ab}\;,\;\;\;\pi_{\alpha\alpha}=0\;,
\end{equation}
где 
\begin{equation}
  p_{id}=\frac{1}{D}\int m\bc^2f(t,m,\br,\bv)\,d\bv=nT\;,
\end{equation}
давление идеального газа. Часть тензора с нулевой сверткой $\pi_{\ab}$ также называется тензором \emph{вязких напряжений}.

Наконец, введем следующий тензор
\begin{equation}
  Q_{\ab\gamma}=m\int c_\alpha c_\beta c_\gamma f(t,m,\br,\bv)\,d\bv\;,
\end{equation}
или точнее его свертку по двум индексам
\begin{equation}
  q_\alpha=\bq = \frac{1}{2}Q_{\ab\beta} = \int \frac{m\bc^2}{2}c_\alpha f(t,m,\br,\bv)\,d\bv\;,
\end{equation}
который называется вектором \emph{потока тепла}.

Теперь приступим к написанию уравнений переноса для вышеперечисленных макропараметров. Для начала напишем
уравнение переноса для некоторой обобщенной динамической функции $A(t,m,\br,\bv)$. 
Умножим данную функцию на уравнение Больцмана (\ref{eq:Boltzmann_general})
и проинтегрируем по всему пространству скоростей:
\begin{equation}
  \int A\frac{\pd f}{\pd t}\,d\bv + \int A\bv\frac{\pd f}{\pd\br}\,d\bv+\int A\bw\frac{\pd f}{\pd\bv} 
  = \int AI_c(f,f')\,d\bv = \left\langle\frac{\pd A}{\pd t}\right\rangle_c\;,
\end{equation}
где правая часть показывает среднее изменение динамической функции по времени за счет столкновений.
Далее можно написать
\begin{equation}
  \int\left(\frac{\pd}{\pd t}(Af)+\frac{\pd}{\pd\br}(Af\bv)+\frac{\pd}{\pd\bv}(Af\bw)-
  f\left[\frac{\pd A}{\pd t}+\bv\frac{\pd A}{\pd\br}+\bw\frac{\pd A}{\pd\bv}\right]\right)
  \,d\bv=\left\langle\frac{\pd A}{\pd t}\right\rangle_c\;.
\end{equation}
Третий член в данном уравнении можно переписать используя теорему Гаусса
 \begin{equation}
   \int\frac{\pd}{\pd\bv}(Af\bw)\,d\bv=\oint Af\bw\cdot d\bm{\sigma} = 0\;.
 \end{equation}
Здесь, интегрирование по всему пространству скоростей заменено на интегрирование по контуру вокруг этого пространства, где
$v\to\pm\infty$. Однако функция распределения $f$ обращается в нуль в этом пределе по своей природе. Таким образом,
мы видим что этот интеграл исчезает. В конечном итоге у нас остается уравнение переноса в следующем виде:
\begin{equation}\label{eq:transport_equation_general}
  \frac{\pd}{\pd t}\int Af\,d\bv+\frac{\pd}{\pd\br}\int Af\bv\,d\bv
  -\int f\left[\frac{\pd A}{\pd t}+\bv\frac{\pd A}{\pd\bv}+\bw\frac{\pd A}{\pd\bv}\right]\,d\bv
  =\left\langle\frac{\pd A}{\pd t}\right\rangle_c\;.
\end{equation}
Теперь, подставляя вместо $A$ необходимые нам макропараметры системы, мы можем вывести соответствующие уравнения переноса для них.

\section{Перенос массы}
Заменяя в уравнении (\ref{eq:transport_equation_general}) динамическую функцию на массу, $A(t,m,\br,\bv)=m$, получаем
\begin{equation}
  \frac{\pd}{\pd t}\int mf\,d\bv + \frac{\pd}{\pd\br}\int mf\bv\,d\bv = 0\;,
\end{equation}
теперь используя (\ref{eq:mass_density}) и (\ref{eq:momentum_density}), получаем уравнение переноса плотности массы,
или так называемое \emph{уравнение непрерывности}
\begin{equation}
  \frac{\pd\rho}{\pd t}+\frac{\pd\rho}{\pd\br}(\rho\bu)=0\;.
\end{equation}

\section{Перенос импульса}

Теперь вместо динамической функции подставляем импульс частицы 
$A(t,m,\br,\bv)=m\bv=mv_\alpha$, и получаем:
\begin{equation}
  \frac{\pd}{\pd t}\int m\bv f\,d\bv+\frac{\pd}{\pd r_\alpha}\int mfv_\alpha v_\beta\,d\bv-\bw\int mf\,d\bv = 
  \left\langle\frac{\pd(mv_{\alpha})}{\pd t}\right\rangle_c\;.
\end{equation}
Правая часть этого уравнения показывает изменение среднего импульса системы в целом, однако по закону сохранения импульса,
оно равняется нулю. Подставляя выражения соответствующих макропараметров, получаем:
\begin{equation}
  \frac{\pd(\rho\bu)}{\pd t} + \frac{\pd\Pi_{\ab}}{\pd r_\alpha} = \rho\bw\;,
\end{equation}
где правая часть $\rho\bw=\bm{F}_{ext}$ -- плотность внешних сил действующих на участок системы. В нашем случае
внешней силой является гравитационное воздействие планеты, которое записывается как $\bw=-\cfrac{\pd U(r)}{\pd\br}$,
где $r$ -- расстояние до центра планеты. Теперь разделяя тензор напряжений получаем уравнение переноса импульса
\begin{equation}
  \frac{\pd(\rho u_\alpha)}{\pd t} + \frac{\pd}{\pd r_\beta}(\rho u_\alpha u_\beta) = 
  -\frac{\pd\left(nT\right)}{\pd r_\alpha} - \frac{\pd\pi_{\ab}}{\pd r_\beta} + \rho w_\alpha\;.
\end{equation}

\section{Перенос энергии}

Теперь подставим функцию кинетической энергии $A(t,m,\br,\bv)=\cfrac{m\bv^2}{2}$ и запишем уравнение переноса
\begin{equation}
  \begin{split}
    &\frac{1}{2}\frac{\pd}{\pd t}\int m\left(\bc^2+2c_\alpha u_\alpha + \bu^2\right)f\,d\bv
    + \frac{1}{2}\frac{\pd}{\pd\br}\int m\left(\bc^2+2c_\alpha u_\alpha + \bu^2\right)v_\beta f\,d\bv-\\
    &-w_\beta\int mfv_\alpha\frac{\pd v_\alpha}{\pd v_\beta}\,d\bv=\left\langle\frac{\pd}{\pd t}\left(\frac{m\bv^2}{2}\right)\right\rangle_c=
    -T\xi(t,m,\br,T)\;.\\
  \end{split}
\end{equation}
По причине диссипативной природы гранулярных газов, закон сохранения энергии не выполняется, и соответственно правая часть уравнения показывает
среднее изменение энергии за счет столкновений. Здесь мы ввели положительную функцию $\xi(t,m,\br,T)$, которая отвечает за скорость
охлаждения газа. Точный вид этой функции мы выведем позднее. Продвигаясь далее записываем:
\begin{equation}
  \begin{split}
    &\frac{\pd}{\pd t}\int\frac{m\bc^2}{2}f\,d\bv+\frac{\pd}{\pd t}\int\frac{m\bu^2}{2}f\,d\bv+
    \frac{\pd}{\pd r_\beta}\int\frac{m\bc^2}{2}v_\beta f\,d\bv + \frac{\pd}{\pd r_\beta}\int\frac{m\bu^2}{2}v_\beta f\,d\bv + \\
    &+\frac{\pd}{\pd r_\beta}\int mc_\alpha u_\alpha v_\beta f\,d\bv = \delta_{\ab}w_\beta\rho u_\alpha - T\xi = u_\alpha\cdot\rho w_\alpha - T\xi\;,
  \end{split}
\end{equation}
где мы использовали условие $\int c_if\,d\bv=0$. Выражая через макропараметры, получаем:
\begin{equation}
  \begin{split}
    &\frac{\pd}{\pd t}\left(\frac{D}{2}nT+\frac{\rho\bu^2}{2}\right)+\frac{\pd}{\pd r_\beta}\int\frac{m\bc^2}{2}(c_\beta+u_\beta)\,d\bv +
    \frac{\pd}{\pd r_\beta}\int mfu_\alpha c_\alpha(c_\beta+u_\beta)\,d\bv +\\
    &+\frac{\pd}{\pd r_\beta}\left(\frac{\rho\bu^2}{2}u_\beta\right)=\rho\bw\cdot\bu-T\xi\;,
  \end{split}
\end{equation}
\begin{equation}
  \begin{split}
    &\frac{\pd}{\pd t}\left(\frac{D}{2}nT+\frac{\rho\bu^2}{2}\right)+\frac{\pd q_\alpha}{\pd r_\alpha}+
    \frac{\pd}{\pd r_\alpha}\left(\frac{D}{2}u_\alpha nT\right)+\frac{\pd}{\pd r_\alpha}u_\beta\int mfc_\alpha c_\beta\,d\bv+\\
    &+\frac{\pd}{\pd r_\alpha}\left(\frac{\rho\bu^2}{2}u_\alpha\right)=\rho\bw\cdot\bu-T\xi\;,
  \end{split}
\end{equation}
и используя выражение для тензора внутренних напряжений, получаем:
\begin{equation}
  \frac{\pd}{\pd t}\left(\frac{D}{2}nT+\frac{\rho\bu^2}{2}\right)+
  \frac{\pd}{\pd r_\alpha}\left(\frac{D}{2}u_\alpha nT+\frac{\rho\bu^2}{2}u_\alpha+\delta_{\ab}u_\beta nT+\pi_{\ab}u_\beta+q_\alpha\right)=
  \rho\bw\cdot\bu-T\xi\;,
\end{equation}
и в итоге получаем уравнение переноса энергии:
\begin{equation}
  \frac{\pd}{\pd t}\left(\frac{D}{2}nT+\frac{\rho\bu^2}{2}\right)+
  \frac{\pd}{\pd r_\alpha}u_\alpha\left(\frac{D+2}{2}nT+\frac{\rho\bu^2}{2}\right)+\frac{\pd}{\pd r_\alpha}(\pi_{\ab}u_\beta)
  +\frac{\pd q_\alpha}{\pd r_\alpha}=\rho\bw\cdot\bu-T\xi\;.
\end{equation}

\section{Вектор потока тепла и тензор вязких напряжений}

Во время вывода уравнений переноса, мы получили два неизвестных нам параметра, а именно $q_\alpha$ -- вектор потока тепла,
и $\pi_{\ab}$ -- тензор вязких напряжений. Выше, во время гидродинамического перехода, мы ввели некий размер в системе 
$\ell\ll L$. Если теперь рассмотреть малый параметр $x=\ell/L\ll 1$ вокруг которого разложим все макропараметры в ряд, то
в нулевом приближении оба параметра $q_\alpha$ и $\pi_{\ab}$ исчезают, так как в этом приближении мы имеем дело с идеальной
жидкостью. В линейном приближении по этому параметру, получаем $q_\alpha,\,\pi_{\ab}\sim x$, и они получают следующий вид
\begin{equation}
  \begin{split}
    \pi_{\ab} &= -\nu\left(\frac{\pd u_\alpha}{\pd r_\beta}+\frac{\pd u_\beta}{\pd r_\alpha}
    -\frac{2}{D}\delta_{\ab}\frac{\pd u_\beta}{\pd r_\alpha}\right)\;,\\
    q_\alpha &= -\lambda\,\mbox{grad}\,T = -\lambda\frac{\pd T}{\pd r_\alpha}\;,
  \end{split}
\end{equation}
где $\nu$ -- коэффициент вязкости и $\lambda$ -- коэффициент теплопроводности.