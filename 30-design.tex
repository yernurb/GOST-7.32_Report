\chapter{Гидродинамическое описание системы}
\label{cha:design}

Перейдем теперь к более грубому, гидродинамическому описанию системы. Для начала нужно определить возможность такого перехода
в рассматриваемой нами системе. Через $L$ обозначим линейный размер всей системы в целом. Для кинетического описания нам было достаточно
что линейные размеры частиц $\sigma$ были намного меньше размера всей системы, т.е. выполнение условия $\sigma\ll L$. Соответственно
все кинетические уравнения пишутся на уровне детализации $\sigma$. Однако гидродинамическое описание происходит на другом уровне,
который мы обозначим $\ell$, и который удовлетворяет условию $\sigma\ll\ell\ll L$. Если мы можем вести изучение системы в таком масштабе,
то можно говорить что мы рассматриваем систему в гидродинамическом приближении. Линейные размеры колец Сатурна, т.е. ширина колец,
растягивается примерно на $L\sim 66 000$ км, в то время как размеры самих частиц варьируются в пределах $\sigma\sim 10^{-2}\div 10^2$ см.
Отсюда хорошо видно что можно подобрать такое значение $\ell\sim 1\div 2$ км, в пределах которого гидродинамическое описание системы
будет вполне оправдано. Таким образом, в дальнейшем мы можем говорить что в пределах $\ell$ вокруг координаты $\br$ находится достаточно большое
количество частиц, по которым можно определить макропараметры системы в зависимости от самой координаты $\br$. Перейдем к непосредственным 
определениям самих макропараметров, необходимых для полного гидродинамического описания системы, и написанию уравнений переноса
этих параметров.

\section{Уравнения переноса}

Зная функцию распределения, можно вводить пространственно распределенные параметры, описывающие систему в целом, а не через отдельно
взятые частицы. Такие параметры называются макропараметрами системы. Мы будем вводить их как моменты вектора скорости частиц $\bv$. Так,
нулевой момент, как мы уже ранее определили, дает нам функцию числовой плотности:
\begin{equation}
  n(t,m,\br) = \int f(t,m,\br,\bv)\,d\bv\;,
\end{equation}
либо функцию плотности масс
\begin{equation}\label{eq:mass_density}
  \rho(t,m,\br) = mn(t,m,\br) = \int mf(t,m,\br,\bv)\,d\bv\;.
\end{equation}
Данная функция читается так: $\rho(t,m,\br)$ -- это масса в единичном объеме участка кольца в координате $\br$ во время $t$, при этом 
масса вещества данного участка кольца равна $m$. Остальные параметры имеют схожий физический смысл.

Первый момент по скорости дает нам функцию плотности импульса
\begin{equation}\label{eq:momentum_density}
  \rho\bu(t,m,\br)=\int m\bv f(t,m,\br,\bv)\,d\bv\;,
\end{equation}
где $\bu$ -- дает нам среднюю скорость участка кольца. Введем понятие локальной скорости частиц
\begin{equation}\label{eq:local_velocity}
  \bc = \bv-\bu(t,m,\br)\;,
\end{equation}
которая показывает скорость (хаотического движения) частиц в системе отсчета движущейся вместе с участком кольца со скоростью $\bu$.
Таким образом введем понятие \emph{гранулярной температуры} системы, по аналогии с термодинамической температурой как второй момент
по скорости
\begin{equation}\label{eq:granular_temperature}
  \frac{D}{2}nT(t,m,\br) = \int\frac{m\bc^2}{2}f(t,m,\br,\bv)\,d\bv\;,
\end{equation}
где $D=3$ -- для трехмерной системы, однако мы будем рассматривать двумерную систему $D=2$.
Нам также понадобятся и другие моменты по скорости, которые уже являются тензорными величинами. Во первых, это
тензор \emph{напряжения}:
\begin{equation}
  \Pi_{\ab}(t,m,\br)=m\int v_\alpha v_\beta f(t,m,\br,\bv)\,d\bv\;,
\end{equation}
который можно разделить на две части используя (\ref{eq:local_velocity})
\begin{equation}
  \Pi_{\ab}(t,m,\br)=\rho u_\alpha u_\beta+P_{\ab}\;.
\end{equation}
Первая часть является динамической частью тензора напряжения, а вторая часть называется тензором \emph{внутренних напряжений}:
\begin{equation}
  P_{\ab}(t,m,\br)=m\int c_\alpha c_\beta f(t,m,\br,\bv)\,d\bv\;.
\end{equation}
Разделяя далее данный тензор на часть с нулевой сверткой и на диагональную часть, получаем:
\begin{equation}
  P_{\ab} = \delta_{\ab}p_{id}+\pi_{\ab}\;,\;\;\;\pi_{\alpha\alpha}=0\;,
\end{equation}
где 
\begin{equation}
  p_{id}=\frac{1}{D}\int m\bc^2f(t,m,\br,\bv)\,d\bv=nT\;,
\end{equation}
давление идеального газа. Часть тензора с нулевой сверткой $\pi_{\ab}$ также называется тензором \emph{вязких напряжений}.

Наконец, введем следующий тензор
\begin{equation}
  Q_{\ab\gamma}=m\int c_\alpha c_\beta c_\gamma f(t,m,\br,\bv)\,d\bv\;,
\end{equation}
или точнее его свертку по двум индексам
\begin{equation}
  q_\alpha=\bq = \frac{1}{2}Q_{\ab\beta} = \int \frac{m\bc^2}{2}c_\alpha f(t,m,\br,\bv)\,d\bv\;,
\end{equation}
который называется вектором \emph{потока тепла}.

Теперь приступим к написанию уравнений переноса для вышеперечисленных макропараметров. Для начала напишем
уравнение переноса для некоторой обобщенной динамической функции $A(t,m,\br,\bv)$. 
Умножим данную функцию на уравнение Больцмана (\ref{eq:Boltzmann_general})
и проинтегрируем по всему пространству скоростей:
\begin{equation}
  \int A\frac{\pd f}{\pd t}\,d\bv + \int A\bv\frac{\pd f}{\pd\br}\,d\bv+\int A\bw\frac{\pd f}{\pd\bv} 
  = \int AI_c(f,f')\,d\bv = \left\langle\frac{\pd A}{\pd t}\right\rangle_c\;,
\end{equation}
где правая часть показывает среднее изменение динамической функции по времени за счет столкновений.
Далее можно написать
\begin{equation}
  \int\left(\frac{\pd}{\pd t}(Af)+\frac{\pd}{\pd\br}(Af\bv)+\frac{\pd}{\pd\bv}(Af\bw)-
  f\left[\frac{\pd A}{\pd t}+\bv\frac{\pd A}{\pd\br}+\bw\frac{\pd A}{\pd\bv}\right]\right)
  \,d\bv=\left\langle\frac{\pd A}{\pd t}\right\rangle_c\;.
\end{equation}
Третий член в данном уравнении можно переписать используя теорему Гаусса
 \begin{equation}
   \int\frac{\pd}{\pd\bv}(Af\bw)\,d\bv=\oint Af\bw\cdot d\bm{\sigma} = 0\;.
 \end{equation}
Здесь, интегрирование по всему пространству скоростей заменено на интегрирование по контуру вокруг этого пространства, где
$v\to\pm\infty$. Однако функция распределения $f$ обращается в нуль в этом пределе по своей природе. Таким образом,
мы видим что этот интеграл исчезает. В конечном итоге у нас остается уравнение переноса в следующем виде:
\begin{equation}\label{eq:transport_equation_general}
  \frac{\pd}{\pd t}\int Af\,d\bv+\frac{\pd}{\pd\br}\int Af\bv\,d\bv
  -\int f\left[\frac{\pd A}{\pd t}+\bv\frac{\pd A}{\pd\bv}+\bw\frac{\pd A}{\pd\bv}\right]\,d\bv
  =\left\langle\frac{\pd A}{\pd t}\right\rangle_c\;.
\end{equation}
Теперь, подставляя вместо $A$ необходимые нам макропараметры системы, мы можем вывести соответствующие уравнения переноса для них.

\subsection{Перенос массы}
Заменяя в уравнении (\ref{eq:transport_equation_general}) динамическую функцию на массу, $A(t,m,\br,\bv)=m$, получаем
\begin{equation}
  \frac{\pd}{\pd t}\int mf\,d\bv + \frac{\pd}{\pd\br}\int mf\bv\,d\bv = 0\;,
\end{equation}
теперь используя (\ref{eq:mass_density}) и (\ref{eq:momentum_density}), получаем уравнение переноса плотности массы,
или так называемое \emph{уравнение непрерывности}
\begin{equation}
  \frac{\pd\rho}{\pd t}+\frac{\pd}{\pd\br}(\rho\bu)=0\;.
\end{equation}

\subsection{Перенос импульса}

Теперь вместо динамической функции подставляем импульс частицы 
$A(t,m,\br,\bv)=m\bv=mv_\alpha$, и получаем:
\begin{equation}
  \frac{\pd}{\pd t}\int m\bv f\,d\bv+\frac{\pd}{\pd r_\alpha}\int mfv_\alpha v_\beta\,d\bv-\bw\int mf\,d\bv = 
  \left\langle\frac{\pd(mv_{\alpha})}{\pd t}\right\rangle_c\;.
\end{equation}
Правая часть этого уравнения показывает изменение среднего импульса системы в целом, однако по закону сохранения импульса,
оно равняется нулю. Подставляя выражения соответствующих макропараметров, получаем:
\begin{equation}
  \frac{\pd(\rho\bu)}{\pd t} + \frac{\pd\Pi_{\ab}}{\pd r_\alpha} = \rho\bw\;,
\end{equation}
где правая часть $\rho\bw=\bm{F}_{ext}$ -- плотность внешних сил действующих на участок системы. В нашем случае
внешней силой является гравитационное воздействие планеты, которое записывается как $\bw=-\cfrac{\pd U(r)}{\pd\br}$,
где $r$ -- расстояние до центра планеты. Теперь разделяя тензор напряжений получаем уравнение переноса импульса
\begin{equation}
  \frac{\pd(\rho u_\alpha)}{\pd t} + \frac{\pd}{\pd r_\beta}(\rho u_\alpha u_\beta) = 
  -\frac{\pd\left(nT\right)}{\pd r_\alpha} - \frac{\pd\pi_{\ab}}{\pd r_\beta} + \rho w_\alpha\;.
\end{equation}

\subsection{Перенос энергии}

Теперь подставим функцию кинетической энергии $A(t,m,\br,\bv)=\cfrac{m\bv^2}{2}$ и запишем уравнение переноса
\begin{equation}
  \begin{split}
    &\frac{1}{2}\frac{\pd}{\pd t}\int m\left(\bc^2+2c_\alpha u_\alpha + \bu^2\right)f\,d\bv
    + \frac{1}{2}\frac{\pd}{\pd\br}\int m\left(\bc^2+2c_\alpha u_\alpha + \bu^2\right)v_\beta f\,d\bv-\\
    &-w_\beta\int mfv_\alpha\frac{\pd v_\alpha}{\pd v_\beta}\,d\bv=\left\langle\frac{\pd}{\pd t}\left(\frac{m\bv^2}{2}\right)\right\rangle_c=
    -T\xi(t,m,\br,T)\;.\\
  \end{split}
\end{equation}
По причине диссипативной природы гранулярных газов, закон сохранения энергии не выполняется, и соответственно правая часть уравнения показывает
среднее изменение энергии за счет столкновений. Здесь мы ввели положительную функцию $\xi(t,m,\br,T)$, которая отвечает за скорость
охлаждения газа. Точный вид этой функции мы выведем позднее. Продвигаясь далее записываем:
\begin{equation}
  \begin{split}
    &\frac{\pd}{\pd t}\int\frac{m\bc^2}{2}f\,d\bv+\frac{\pd}{\pd t}\int\frac{m\bu^2}{2}f\,d\bv+
    \frac{\pd}{\pd r_\beta}\int\frac{m\bc^2}{2}v_\beta f\,d\bv + \frac{\pd}{\pd r_\beta}\int\frac{m\bu^2}{2}v_\beta f\,d\bv + \\
    &+\frac{\pd}{\pd r_\beta}\int mc_\alpha u_\alpha v_\beta f\,d\bv = \delta_{\ab}w_\beta\rho u_\alpha - T\xi = u_\alpha\cdot\rho w_\alpha - T\xi\;,
  \end{split}
\end{equation}
где мы использовали условие $\int c_if\,d\bv=0$. Выражая через макропараметры, получаем:
\begin{equation}
  \begin{split}
    &\frac{\pd}{\pd t}\left(\frac{D}{2}nT+\frac{\rho\bu^2}{2}\right)+\frac{\pd}{\pd r_\beta}\int\frac{m\bc^2}{2}(c_\beta+u_\beta)\,d\bv +
    \frac{\pd}{\pd r_\beta}\int mfu_\alpha c_\alpha(c_\beta+u_\beta)\,d\bv +\\
    &+\frac{\pd}{\pd r_\beta}\left(\frac{\rho\bu^2}{2}u_\beta\right)=\rho\bw\cdot\bu-T\xi\;,
  \end{split}
\end{equation}
\begin{equation}
  \begin{split}
    &\frac{\pd}{\pd t}\left(\frac{D}{2}nT+\frac{\rho\bu^2}{2}\right)+\frac{\pd q_\alpha}{\pd r_\alpha}+
    \frac{\pd}{\pd r_\alpha}\left(\frac{D}{2}u_\alpha nT\right)+\frac{\pd}{\pd r_\alpha}u_\beta\int mfc_\alpha c_\beta\,d\bv+\\
    &+\frac{\pd}{\pd r_\alpha}\left(\frac{\rho\bu^2}{2}u_\alpha\right)=\rho\bw\cdot\bu-T\xi\;,
  \end{split}
\end{equation}
и используя выражение для тензора внутренних напряжений, получаем:
\begin{equation}
  \frac{\pd}{\pd t}\left(\frac{D}{2}nT+\frac{\rho\bu^2}{2}\right)+
  \frac{\pd}{\pd r_\alpha}\left(\frac{D}{2}u_\alpha nT+\frac{\rho\bu^2}{2}u_\alpha+\delta_{\ab}u_\beta nT+\pi_{\ab}u_\beta+q_\alpha\right)=
  \rho\bw\cdot\bu-T\xi\;,
\end{equation}
и в итоге получаем уравнение переноса энергии:
\begin{equation}
  \frac{\pd}{\pd t}\left(\frac{D}{2}nT+\frac{\rho\bu^2}{2}\right)+
  \frac{\pd}{\pd r_\alpha}u_\alpha\left(\frac{D+2}{2}nT+\frac{\rho\bu^2}{2}\right)+\frac{\pd}{\pd r_\alpha}(\pi_{\ab}u_\beta)
  +\frac{\pd q_\alpha}{\pd r_\alpha}=\rho\bw\cdot\bu-T\xi\;.
\end{equation}

\subsection{Вектор потока тепла и тензор вязких напряжений}

Во время вывода уравнений переноса, мы получили два неизвестных нам параметра, а именно $q_\alpha$ -- вектор потока тепла,
и $\pi_{\ab}$ -- тензор вязких напряжений. Выше, во время гидродинамического перехода, мы ввели некий размер в системе 
$\ell\ll L$. Если теперь рассмотреть малый параметр $x=\ell/L\ll 1$ вокруг которого разложим все макропараметры в ряд, то
в нулевом приближении оба параметра $q_\alpha$ и $\pi_{\ab}$ исчезают, так как в этом приближении мы имеем дело с идеальной
жидкостью. В линейном приближении по этому параметру, получаем $q_\alpha,\,\pi_{\ab}\sim x$, и они получают следующий вид
\begin{equation}\label{eq:viscosity_coeff}
  \begin{split}
    \pi_{\ab} &= -\nu\left(\frac{\pd u_\alpha}{\pd r_\beta}+\frac{\pd u_\beta}{\pd r_\alpha}
    -\frac{2}{D}\delta_{\ab}\frac{\pd u_\beta}{\pd r_\alpha}\right)\;,\\
    q_\alpha &= -\lambda\,\mbox{grad}\,T = -\lambda\frac{\pd T}{\pd r_\alpha}\;,
  \end{split}
\end{equation}
где $\nu$ -- коэффициент вязкости и $\lambda$ -- коэффициент теплопроводности.

\section{Скорость охлаждения системы}

Выше, при выводе уравнения переноса энергии, мы получили некую функцию $\xi(t,m,\br,T)$ которая показывает охлаждение
гранулярного газа за счет диссипативных столкновений. В общем виде эта функция определяется через интеграл столкновений,
как среднее изменение кинетической энергии:
\begin{equation}
  -T_i\xi(t,m_i,\br,T_i) = \int\frac{m_i\bv^2}{2}I_c(t,m_i,\br,\bv)\,d\bv\;.
\end{equation}
Здесь мы учли тот факт, что температуры отдельных подсистем могут быть отличны друг от друга, и показали эту зависимость
через индекс $i$. Используя свойство интеграла столкновений (\ref{eq:collision_integral_dynamic_function}), выпишем заново 
уравнение эволюции энергии системы за счет столкновений (\ref{eq:energy_collision_evolution})б в следующем виде:
\begin{equation}
  \begin{split}
    -T_i\xi(t,m_i,\br, T_i) &= 
    \int dm_j\eta(m_j)\sigma^{D-1}_{ij}\int d\bv_i d\bg\int d\bn\Theta(-\bg\cdot\bn)\vert\bg\cdot\bn\vert\times \\
    &\times f(t,m_i,\br,\bv_i)f(t,m_j,\br,\bv_i-\bg)\delta E_i(\bv_i,\bg) \;.
  \end{split}
\end{equation}

До сих пор мы ничего не говорили о виде самой функции распределения $f(t,m,\br,\bv)$, и все выводы получали для общего 
случая. Однако для дальнейшего продвижения нам необходимо указать вид этой функции. Функцию распределения можно представить
как разложение в ряд по малому параметру $x=\ell/L\ll 1$:
\begin{equation}
  f = f^{(0)} + c_1xf^{(1)}+c_2x^2f^{(2)}+\dots\;,
\end{equation}
где $c_i$ -- коэффициенты полина Сонина, $f^{(0)}$ -- нулевое приближение функции распределения, не что иное как функция
распределения Максвелла. Для простоты мы ограничимся этим приближением, так как его достаточно для описания исследуемых нами эффектов.
Таким образом, в дальнейшем принимаем:
\begin{equation}\label{eq:Maxwell_distribution}
  f(t,m_i,\br,\bv_i) = n_i \left(\frac{m_i}{2\pi T_i}\right)^{D/2}\cdot\exp\left\{-\frac{m_i\left(\bv_i-\bu_i\right)^2}{2T_i}\right\}\;,
\end{equation}
где $n_i$ -- числовая плотность частиц. Средняя скорость движения $\bu$ в нашем случае является скоростью на кеплеровской орбите, 
и зависит только от 
расстояния до центра планеты $\br$, и не зависит от массы частицы. Таким образом можно сделать замену переменных:
\begin{equation}
  \bc_i = \bv_i - \bu(\br)\;,\;\;\;d\bc_i=d\bv_i\;,
\end{equation}
и переписать функцию распределения в следующем виде:
\begin{equation}
  f(t,m_i,\br,\bc_i) = n_i\left(\frac{\kp_i}{\pi}\right)^{D/2}\cdot\exp\left(-\kp_ic^2_i\right)\;,
\end{equation}
где $\kp_i=\cfrac{m_i}{2T_i}$, и далее пишем:
\begin{equation}
  f(t,m_i,\br,\bc_i)f(t,m_j,\br,\bc_j)=n_in_j\left(\frac{\kp_i\kp_j}{\pi^2}\right)^{D/2}
  \cdot\exp\left(-\kp_ic^2_i-\kp_jc^2_j\right)\;.
\end{equation}
 Используя $\bv_i-\bv_j=\bc_i-\bc_j$ и соответственно $\bc_j=\bg-\bc_i$, перепишем выражение под экспонентой в следующем виде:
 \begin{equation}
   \begin{split}
     \kp_ic^2_i+\kp_jc^2_j &= \kp_ic^2_i+\kp_j(\bg-\bc_i)^2=\kp_ic^2_i+\kp_j\left(g^2+c^2_i-2\bg\cdot\bc_i\right)=\\
     &=(\kp_i+\kp_j)c^2_i+\kp_jg^2-2\kp_jgc_i\cos\gamma\;,
   \end{split}
 \end{equation}
где $\gamma$ -- угол между векторами $\bg$ и $\bc_i$. Теперь, среднее изменение некоторой динамической функции по времени
за счет столкновений можно представить с следующем виде:
\begin{equation}
  \begin{split}
    \frac{d}{dt}\langle\psi_i(\bc_i)\rangle &= \int dm_jn_in_j\eta(m_j)
    \sigma^{D-1}_{ij}\left(\frac{\kp_i\kp_j}{\pi^2}\right)^{D/2}\times\\
    &\times\int d\bc_id\bg\int d\bn\Theta(-\bg\cdot\bn)\vert\bg\cdot\bn\vert\times\\
    &\times\exp\left(-(\kp_i+\kp_j)c^2_i-\kp_jg^2+2\kp_jgc_i\cos\gamma\right)\Delta\psi_i(\bg,\bc_i)\;.
  \end{split}
\end{equation}


\subsection{Анализ диссипации для диска}
Для дальнейшего анализа, используем то обстоятельство, что кольца Сатурна являют чрезвычайно тонким и плоским образованием.
При радиальной ширине около $L\sim 66 000$ км, и азимутальной протяженности более полумиллиона километров, имеет толщину 
порядка нескольких метров. Данный факт делает кольцо Сатурна самым тонким природным образованием в солнечной системе. 
Нас же интересует радиальное распределение макропараметров, а не их распределение по толщине. Поэтому, во всех дальнейших
анализах примем рассматриваемую нами систему двумерной $D=2$. Полученные результаты будут отличаться от трехмерной системы
$D=3$ лишь некоторыми числовыми коэффициентами. Таким образом, интегрирование по сечению $d\bn\Theta(-\bg\cdot\bn)$ приводит
к интегралу по полуокружности, и дает нам $\pi$. Обозначая через $\theta$ угол между векторами $\bg$ и $\bn$, можно написать
$\vert\bg\cdot\bn\vert=g\cos\theta$, где $\theta$ меняется между $-\pi/2$ и $+\pi/2$. Перейдя в полярные координаты, получаем
$d\bc_id\bg=gc_i dg dc_id\theta d\gamma$, где $\gamma\in[0\div 2\pi]$. Подставляя все это в интеграл столкновений, пишем:
\begin{equation}
  \begin{split}
    \frac{d}{dt}\langle\psi_i(\bc_i)\rangle &=\frac{n_i\kp_i}{\pi}\int dm_jn_j\kp_j\eta(m_j)\sigma_{ij}
    \int dgdc_i\int_{-\pi/2}^{\pi/2}d\theta\int_0^{2\pi}d\gamma g^2c_i\cos\theta\times \\
    &\times\exp\left(-(\kp_i+\kp_j)c^2_i-\kp_jg^2+2\kp_jgc_i\cos\gamma\right)\Delta\psi_i(\bg,\bc_i)\;.
  \end{split}
\end{equation}
Теперь вместо $\psi_i(\bc_i)$ подставим (\ref{eq:delta_E_v}) и получаем:
\begin{equation}
  \begin{split}
    \left\langle\frac{dE_i}{dt}\right\rangle_c &= \frac{n_i\kp_i}{\pi}\int dm_jn_j\kp_j\eta(m_j)\sigma_{ij}
    \int dgdc_i g^2c_i\exp\left(-(\kp_i+\kp_j)c^2_i-\kp_jg^2\right)\times\\
    &\times\int_{-\pi/2}^{\pi/2}d\theta\cos\theta\int_{0}^{2\pi}d\gamma\exp(2\kp_jgc_i\cos\gamma)\times \\
    &\times\left(-\mu(1+\eps)gc_i\cos\theta\cos(\gamma-\theta)+\frac{\mu^2}{2m_i}(1+\eps)^2g^2\cos^2\theta\right)\;,
  \end{split}
\end{equation}
здесь мы использовали $\bc_i\cdot\bn=c_i\cos(\gamma-\theta)$. Рассмотрим отдельно угловые интегралы:
\begin{equation}
  \begin{split}
    S_{\theta\gamma,1} &= -\mu(1+\eps)gc_i\int_{-\pi/2}^{\pi/2}d\theta\cos^2\theta\int_{0}^{2\pi}d\gamma\cos(\theta-\gamma)\exp(R\cos\gamma)\;,\\
    S_{\theta\gamma,2} &= \frac{\mu^2g^2}{2m_i}(1+\eps)^2\int_{-\pi/2}^{\pi/2}d\theta\cos^3\theta\int_0^{2\pi}d\gamma\exp(R\cos\gamma)\;,
  \end{split}
\end{equation}
где $R=2\kp_jgc_i\geq 0$. Разложим первый интеграл по $\gamma$:
\begin{equation}
  \begin{split}
    S_{\gamma,1} &= \cos\theta\int_{0}^{2\pi}\cos\gamma\exp(R\cos\gamma)\,d\gamma+\sin\theta\int_{0}^{2\pi}\sin\gamma\exp(R\cos\gamma)\,d\gamma=\\
    &=\cos\theta\int_{0}^{2\pi}\cos\gamma\exp(R\cos\gamma)\,d\gamma = 2\cos\theta\int_{0}^{\pi}\cos\gamma\exp(R\cos\gamma)\,d\gamma\;.
  \end{split}
\end{equation}
Здесь функция под первым интегралом четная, поэтому ее можно разделить пополам, а функция под вторым интегралом нечетная, и поэтому исчезает.
Таким образом у нас остаются два интеграла по $\gamma$, которые не интегрируются в квадратурах, и представляются в виде модифицированных функций 
Бесселя:
\begin{equation}
  \begin{split}
    S_{\gamma,1} &= \int_{0}^{\pi}\cos\gamma\exp(R\cos\gamma)\,d\gamma = \pi I_1(R)\;,\\
    S_{\gamma,2} &= \int_{0}^{\pi}\exp(R\cos\gamma)\,d\gamma = \pi I_0(R)\;,
  \end{split}
\end{equation}
где
\begin{equation}
  I_\nu(x) = \frac{1}{\pi}\int_{0}^{\pi}e^{x\cos t}\cos(\nu t)\,dt-\frac{\sin(\pi\nu)}{\pi}\int_{0}^{\infty}e^{-x\cosh t-\nu t}\,dt\;,
\end{equation}
называется \emph{модифицированной функцией Бесселя}. Для значений параметра $\nu=0,\,1$ мы получаем наши интегралы по $\gamma$:
\begin{equation}
  \begin{split}
    S_{\theta\gamma,1} &= -\frac{8}{3}\pi\mu(1+\eps)gc_iI_1(R)\;,\\
    S_{\theta\gamma,2} &= \frac{4\pi\mu^2g^2}{3m_i}(1+\eps)^2I_0(R)\;,
  \end{split}
\end{equation}
где интеграл по $\theta$ элементарен:
\begin{equation}
  \begin{split}    
    \int_{-\pi/2}^{\pi/2} \cos^3\theta\,d\theta &= \frac{4}{3}\;.
  \end{split}
\end{equation}
Теперь исходный интеграл записывается в виде:
\begin{equation}
  \begin{split}
    \left\langle\frac{dE_i}{dt}\right\rangle_c &= \frac{4n_i\kp_i}{3}(1+\eps)\int dm_j\mu n_j\kp_j\eta(m_j)\sigma_{ij}
    \int dgdc_i g^3c_i\times\\
    &\times\exp(-(\kp_i+\kp_j)c^2_i-\kp_jg^2)\times\\
    &\times\left((1+\eps)\frac{\mu}{m_i}gI_0(R)-2c_iI_1(R)\right)\;,
  \end{split}
\end{equation}
или в более развернутом виде:
\begin{equation}
  \begin{split}
    \left\langle\frac{dE_i}{dt}\right\rangle_c &= \frac{4n_i\kp_i}{3}(1+\eps)\int dm_j\mu n_j\kp_j\eta(m_j)\sigma_{ij}
    \int_{0}^{\infty}dg\,g^3e^{-\kp_jg^2}\times\\
    &\times\int_{0}^{\infty}dc_i\,c_i\exp\left(-(\kp_i+\kp_j)c^2_i\right)\times\\
    &\times\left\{(1+\eps)\frac{\mu}{m_i}gI_0(2\kp_jgc_i)-2c_iI_1(2\kp_jgc_i)\right\}\;.
  \end{split}
\end{equation}
Интегралы по функциям Бесселя могут быть рассчитаны с помощью следующих формул:
\begin{equation}
  \begin{split}
    \int_{0}^{\infty}&x^{\alpha-1}e^{-px^2}I_{\nu}(cx)\,dx = A^{\alpha}_{\nu}\;,\;\;\;\left[\Re(p),\;\Re(\alpha+\nu)>0,\;\vert\arg c\vert<\pi\right]\;,\\
    &A^{\alpha}_{\nu}=2^{-\nu-1}c^{\nu}p^{-(\alpha+\nu)/2}\cdot\frac{\Gamma((\alpha+\nu)/2)}{\Gamma(\nu+1)}\cdot
    \mbox{}_{1}F_{1}\left(\frac{\alpha+\nu}{2};\;\nu+1;\;\frac{c^2}{4p}\right)\;,
  \end{split}
\end{equation}
где 
\begin{equation}
  \mbox{}_{1}F_{1}(a;\,b;\,z) = \sum_{n=0}^{\infty}\frac{a^{(a)}z^n}{b^{(n)}n!}\;,
\end{equation}
называется \emph{вырожденной гипергеометрической функцией}. Для наших интегралов, нам нужны два специальных случая:
\begin{equation}
  \begin{split}
    \int_{0}^{\infty}xe^{-px^2}I_0(cx)\,dx &= A^{2}_{0}\;,\\
    \int_{0}^{\infty}x^2e^{-px^2}I_1(cx)\,dx &= A^{3}_{1}\;,
  \end{split}
\end{equation}
где
\begin{equation}
  \begin{split}
    p &= \kp_i+\kp_j\;,\\
    c &= 2\kp_jg\;.
  \end{split}
\end{equation}
Если выполняется условие $\alpha=\nu+2$, то интегральная формула может быть сильно упрощена:
\begin{equation}
  A^{\nu+2}_{\nu} = \frac{c^{\nu}}{(2p)^{\nu+1}}\exp\left(\frac{c^2}{4p}\right)\;.
\end{equation}
В итоге получаем:
\begin{equation}
  \begin{split}
    \int_{0}^{\infty}xe^{-px^2}I_0(cx)\,dx &= \frac{1}{2p}\exp\left(\frac{c^2}{4p}\right)\;,\\
    \int_{0}^{\infty}x^2e^{-px^2}I_1(cx)\,dx &= \frac{c}{(2p)^2}\exp\left(\frac{c^2}{4p}\right)\;.
  \end{split}
\end{equation}
Подставляя все выражения на места, и имея ввиду табличный интеграл:
\begin{equation}      
    \int_{0}^{\infty}x^4e^{-ax^2}\,dx = \frac{3}{8a^2}\sqrt{\frac{\pi}{a}}\;,
\end{equation}
получаем:
\begin{equation}  
    \left\langle\frac{dE_i}{dt}\right\rangle_c = \frac{\sqrt{\pi}}{2}\int dm_j\Lambda_{ij}\mu(1+\eps)
    \frac{\kp_i+\kp_j}{\kp_i\kp_j}\sqrt{\frac{\kp_i+\kp_j}{\kp_i\kp_j}}\left(\frac{1+\eps}{2}\frac{\mu}{m_i}
    -\frac{\kp_j}{\kp_i+\kp_j}\right)\;,
\end{equation}
где
\begin{equation}
  \Lambda_{ij} = n_in_j\eta(m_j)\sigma_{ij}\;.
\end{equation}
Обозначая $\eta(m_j)dm_j=d\chi_j$ -- количество частиц подсистемы $j$ в элементарном объеме вокруг $\br$, получаем
\begin{equation}  
    \left\langle\frac{dE_i}{dt}\right\rangle_c = \frac{\sqrt{\pi}}{2}\cdot n_i(1+\eps)\int d\chi_jn_j \sigma_{ij}\mu_{ij}
    \frac{\kp_i+\kp_j}{\kp_i\kp_j}\sqrt{\frac{\kp_i+\kp_j}{\kp_i\kp_j}}\left(\frac{1+\eps}{2}\frac{\mu_{ij}}{m_i}
    -\frac{\kp_j}{\kp_i+\kp_j}\right)\;,    
\end{equation}
Продвигаемся далее. Так как:
\begin{equation}
  \frac{\kp_j}{\kp_i+\kp_j} = \frac{m_jT_i}{m_iT_j+m_jT_i}\;,
\end{equation}
и
\begin{equation}
  \mu_{ij}\cdot\frac{\kp_i+\kp_j}{\kp_i\kp_j} = \frac{m_im_j}{m_i+m_j}\frac{\frac{m_i}{2T_i}+\frac{m_j}{2T_j}}{\frac{m_im_j}{4T_iT_j}}=
  2\cdot\frac{m_iT_j+m_jT_i}{m_i+m_j}\;,
\end{equation}
преобразуем выражение под интегралом
\begin{equation*}
  \begin{split}
    S &= \mu_{ij}(1+\eps)\frac{\kp_i+\kp_j}{\kp_i\kp_j}\times\left(\frac{\mu_{ij}(1+\eps)}{2m_i}-\frac{\kp_j}{\kp_i+\kp_j}\right) = \\
    &= \frac{m_iT_j+m_jT_i}{m_i+m_j}\left(\frac{m_j}{m_i+m_j}(1+\eps)^2-\frac{2(1+\eps)m_jT_i}{m_iT_j+m_jT_i}\right)=\\
    &= \frac{m_j}{(m_i+m_j)^2}\left((m_iT_j+m_jT_i)(1+\eps)^2-2(1+\eps)(m_i+m_j)T_i\right)=\\
    &= \frac{m_j}{(m_i+m_j)^2}\left([(1+\eps)^2T_j-2(1+\eps)T_i]m_i+[(1+\eps)^2-2(1+\eps)]m_jT_i\right)=\\
    &= \frac{m_j}{(m_i+m_j)^2}\left(-\left(1-\eps^2\right)m_jT_i+(1+\eps)[(1+\eps)T_j-2T_i]m_i\right)\;,
  \end{split}
\end{equation*}
далее, имея ввиду что $2=1+\eps+1-\eps$, пишем:
\begin{equation*}
  \begin{split}
    S &= \frac{m_j}{(m_i+m_j)^2}\left(-\left(1-\eps^2\right)m_jT_i+m_i(1+\eps)[(1+\eps)T_j-(1+\eps)T_i-(1-\eps)T_i]\right)=\\
    &= \frac{m_j}{(m_i+m_j)^2}\left(-\left(1-\eps^2\right)m_jT_i+m_i(1+\eps)^2\left(T_j-T_i\right)-m_i\left(1-\eps^2\right)T_i\right) =\\
    &= \frac{m_j}{(m_i+m_j)^2}\left(-\left(1-\eps^2\right)(m_i+m_j)T_i+m_i(1+\eps)^2\left(T_j-T_i\right)\right)=\\
    &= -\left(1-\eps^2\right)\frac{m_j}{m_i+m_j}T_i+(1+\eps)^2\frac{m_im_j}{(m_i+m_j)^2}\left(T_j-T_i\right)\;,
  \end{split}
\end{equation*}
и наконец
\begin{equation}
  S = -\left(1-\eps^2\right)\frac{\mu}{m_i}T_i+(1+\eps)^2\frac{\mu^2}{m_im_j}\left(T_j-T_i\right)\;.
\end{equation}
Вставляя полученное выражение в исходный интеграл, получаем:
\begin{equation}\label{eq:cooling_rate_interm}
  \begin{split}
    \left\langle\frac{dE_i}{dt}\right\rangle_c &= \sqrt{\frac{\pi}{2}}\cdot n_i\int d\chi_jn_j\sigma_{ij}
    \sqrt{\frac{m_iT_j+m_jT_i}{m_im_j}}\times \\
    &\times\left(-\left(1-\eps^2\right)\frac{\mu}{m_i}T_i+(1+\eps)^2\frac{\mu^2}{m_im_j}(T_j-T_i)\right)\;.    
  \end{split}
\end{equation}

\subsection{Усредненные по ансамблю динамические параметры смеси гранулярных газов}
Рассмотрим некоторые динамические параметры смеси гранулярных газов, которые будут необходимы нам для дальнейшего анализа.
Возьмем некоторую функцию зависящую от двухчастичного распределения $\alpha_{ij}$. В общем виде, среднее значение
двухчастичной функции задается следующим выражением:
\begin{equation}
  \begin{split}
    \langle \alpha_{ij}\rangle &= \frac{\int \alpha_{ij}\cdot f(t,m_i,\br,\bc_i)f(t,m_j,\br,\bc_j)\,d\bc_id\bc_j}
    {\int f(t,m_i,\br,\bc_i)f(t,m_j,\br,\bc_j)\,d\bc_id\bc_j} = \\
    &= \frac{1}{n_in_j}\int \alpha_{ij}\cdot f(t,m_i,\br,\bc_i)f(t,m_j,\br,\bc_j)\,d\bc_id\bc_j\;,
  \end{split}
\end{equation}
и принимая во внимание что функцию распределения мы выбрали в виде Максвелловского распределения (\ref{eq:Maxwell_distribution}), и 
следуя такой же схеме рассуждений как и в предыдущей главе, пишем:
\begin{equation}
  \begin{split}
    \langle \alpha_{ij}\rangle &= \frac{\kp_i\kp_j}{\pi^2}\int_{0}^{\infty}dg\,ge^{-\kp_jg^2}\int_{0}^{\infty}dc_i\,c_i\alpha_{ij}e^{-(\kp_i+\kp_j)c^2_i}
    \int_{0}^{2\pi}d\phi\int_{0}^{2\pi}d\gamma\,\exp(2\kp_jgc_i\cos\gamma)=\\
    &=4\kp_i\kp_j\int_{0}^{\infty}dg\,ge^{-\kp_jg^2}\int_{0}^{\infty}dc_i\,c_i\alpha_{ij}I_0(2\kp_jg\cdot c_i)
    \exp\left(-(\kp_i+\kp_j)c^2_i\right)\;,
  \end{split}
\end{equation}
где под $\alpha_{ij}$ мы понимаем некоторую функцию по комбинациям скоростей частиц. Сначала положим $\alpha_{ij}=g_{ij}$, 
и в итоге получаем среднюю скорость столкновений для частиц подсистем $i$ и $j$ для двумерного газа:
\begin{equation}
    \langle g_{ij}\rangle = 4\kp_i\kp_j\int_{0}^{\infty}dg\,g^2e^{-\kp_jg^2}\int_{0}^{\infty}dc_i\,c_i
    I_0(\lambda\cdot c_i)\exp\left(-pc^2_i\right)\;,
\end{equation}
где $p=\kp_i+\kp_j$, $\lambda = 2\kp_j g$. Нам также понадобятся и другие динамические параметры. Это среднее по ансамблю
квадрата скорости столкновения, которое присутствует в выражениях для энергии в системе.
\begin{equation}
  \langle g^2_{ij}\rangle = 4\kp_i\kp_j\int_{0}^{\infty}dg\,g^3e^{-\kp_jg^2}\int_{0}^{\infty}dc_i\,c_i
  I_0(\lambda\cdot c_i)\exp\left(-pc^2_i\right)\;.
\end{equation}
Также запишем следующую функцию $\alpha_{ij} = \bg_{ij}\cdot\bc_i=g_{ij}c_i\cos\gamma$, и получаем:
\begin{equation}
  \langle \bg_{ij}\cdot\bc_i\rangle = 4\kp_i\kp_j\int_{0}^{\infty}dg\,g^2e^{-\kp_jg^2}\int_{0}^{\infty}dc_i\,c^2_i
  I_1(\lambda\cdot c_i)\exp\left(-pc^2_i\right)\;.
\end{equation}
Таким образом:
\begin{equation}
  \begin{split}
    \langle g_{ij}\rangle &= 4\kp_i\kp_j\int_{0}^{\infty}dg\,g^2e^{-\kp_jg^2}\cdot\frac{1}{2p}\exp\left(\frac{\lambda^2}{4p}\right)\;,\\
    \langle g^2_{ij}\rangle &= 4\kp_i\kp_j\int_{0}^{\infty}dg\,g^3e^{-\kp_jg^2}\cdot\frac{1}{2p}\exp\left(\frac{\lambda^2}{4p}\right)\;,\\
    \langle \bg_{ij}\cdot\bc_i\rangle &= 4\kp_i\kp_j\int_{0}^{\infty}dg\,g^2e^{-\kp_jg^2}\cdot\frac{\lambda}{4p^2}\exp\left(\frac{\lambda^2}{4p}\right)\;,
  \end{split}
\end{equation}
или
\begin{equation}
  \begin{split}
    \langle g_{ij}\rangle &= 2R\int_{0}^{\infty}dg\,g^2e^{-Rg^2}=\frac{1}{2}\sqrt{\frac{\pi}{R}}\;,\\
    \langle g^2_{ij}\rangle &= 2R\int_{0}^{\infty}dg\,g^3e^{-Rg^2}=\frac{1}{R}\;,\\
    \langle \bg_{ij}\cdot\bc_i\rangle &= \frac{2R^2}{\kp_i}\int_{0}^{\infty}dg\,g^3e^{-Rg^2}=\frac{1}{\kp_i}\;,
  \end{split}
\end{equation}
где 
\begin{equation}
  R=\frac{\kp_i\kp_j}{\kp_i+\kp_j}=\frac{1}{2}\frac{m_im_j}{m_iT_j+m_jT_i}\;.
\end{equation}
В конечном итоге получаем:
\begin{equation}
  \begin{split}
    \langle g_{ij}\rangle &= \sqrt{\frac{\pi}{2}}\cdot\sqrt{\frac{m_iT_j+m_jT_i}{m_im_j}}\;,\\
    \langle g^2_{ij}\rangle &= 2\cdot\frac{m_iT_j+m_jT_i}{m_im_j}\;,\\
    \langle \bg_{ij}\cdot\bc_i\rangle &= \frac{2T_i}{m_i}\;.
  \end{split}
\end{equation}

Оценим частоту столкновений частицы подсистемы $i$ с частицами подсистемы $j$. Представим что размер частицы $i$ пренебрежимо
мал, и она движется среди других частиц с размерами $\sigma_{ij}=\sigma_i+\sigma_j$, и плотностью $n_j$. В таком случае частота
столкновений равна:
\begin{equation}
  \omega_{ij} = \sigma_{ij}n_j\langle g_{ij}\rangle=\sqrt{\frac{\pi}{2}}\cdot\sigma_{ij}n_j\sqrt{\frac{m_iT_j+m_jT_i}{m_im_j}}\;.
\end{equation}
Теперь посчитаем среднее по ансамблю изменение энергии при столкновениях
(\ref{eq:delta_E}):
\begin{equation}
  \begin{split}
    \langle\delta E_i\rangle &= -\mu(1+\eps)\langle \bg\cdot\bv_C\rangle-\frac{1-\eps^2}{2}\frac{\mu^2}{m_i}\langle g^2\rangle=\\
    &=-\frac{\mu^2}{m_j}(1+\eps)\langle \bg\cdot\bc_i\rangle-\frac{\mu^2}{m_i}(1+\eps)\langle \bg\cdot\bc_j\rangle
    -\frac{1-\eps^2}{2}\frac{\mu^2}{m_i}\langle g^2\rangle = \\
    &=-\frac{\mu^2}{m_j}(1+\eps)\langle \bg\cdot\bc_i\rangle-\frac{\mu^2}{m_i}(1+\eps)\langle \bg\cdot(\bc_i-\bg)\rangle
    -\frac{1-\eps^2}{2}\frac{\mu^2}{m_i}\langle g^2\rangle = \\
    &=-\mu(1+\eps)\langle \bg\cdot\bc_i\rangle+\frac{\mu^2}{m_i}(1+\eps)\langle g^2\rangle
    -\frac{1-\eps^2}{2}\frac{\mu^2}{m_i}\langle g^2\rangle = \\
    &=-\mu(1+\eps)\langle \bg\cdot\bc_i\rangle+\frac{\mu^2}{m_i}\frac{(1+\eps)^2}{2}\langle g^2\rangle \;,
  \end{split}
\end{equation}
далее пишем:
\begin{equation}
  \begin{split}
    \langle\delta E_i\rangle &=-\mu(1+\eps)\frac{2T_i}{m_i}+\frac{\mu^2}{m_i}(1+\eps)^2\cdot\frac{m_iT_j+m_jT_i}{m_im_j}=\\
    &=\frac{\mu}{m_i}(1+\eps)\left((1+\eps)\cdot\frac{m_iT_j+m_jT_i}{m_i+m_j}-2T_i\right)=\\
    &=\frac{\mu}{m_i}\frac{1+\eps}{m_i+m_j}\left[(1+\eps)m_iT_j+(1+\eps)m_jT_i-2m_iT_i-2m_jT_i\right]=\\
    &=\frac{\mu}{m_i}\frac{1+\eps}{m_i+m_j}\left[(1+\eps)m_i(T_j-T_i)-(1-\eps)(m_i+m_j)T_i\right]\;,
  \end{split}
\end{equation}
и в итоге получаем:
\begin{equation}
  \langle\delta E_i\rangle = -\left(1-\eps^2\right)\frac{\mu}{m_i}T_i+(1+\eps)^2\frac{\mu^2}{m_im_j}(T_j-T_i)\;.
\end{equation}
Данное выражение равно выражению под скобкой в уравнении (\ref{eq:cooling_rate_interm}), и в итоге ее можно переписать в
следующем виде:
\begin{equation}
  \left\langle\frac{dE_i}{dt}\right\rangle_c = n_i\int d\chi_j\omega_{ij}\langle\delta E_i\rangle\;.
\end{equation} 
Вспомним что скорость изменения энергии системы за счет столкновений мы выразили через функцию охлаждения $\xi(t,m_i,\br,T_i)$, 
и теперь мы можем написать вид этой функции следующим образом:
\begin{equation}
  \left\langle\frac{dE_i}{dt}\right\rangle_c = -T_i\xi(t,m_i,\br,T_i)\;,
\end{equation} 
или
\begin{equation}\label{eq:cooling_term}
  \xi(t,m_i,\br,T_i)=-\frac{n_i}{T_i}\int d\chi_i\omega_{ij}\left(-A_{ij}T_i+B_{ij}(T_j-T_i)\right)\;,
\end{equation}
где
\begin{equation}
  \begin{split}
    A_{ij} &= \left(1-\eps^2\right)\frac{\mu}{m_i}\;,\\
    B_{ij} &= (1+\eps)^2\frac{\mu^2}{m_im_j}\;.
  \end{split}
\end{equation}
Здесь, параметр $A_{ij}$ показывает скорость охлаждения системы из-за диссипации при столкновениях, 
параметр $B_{ij}$ показывает скорость обмена энергии между подсистемами $i$ и $j$. Данная часть энергии
остается в системе, и не диссипирует.

\subsection{Нормальное решение и приближение по среднему полю}
Рассматриваемая нами система полидисперсных гранулярных газов, после начального релаксационного периода, приходит 
к универсальному, т.н. \emph{нормальному} решению. Смысл этого решения в том, что все зависимости функции
распределения от времени и пространства выражаются только через гидродинамические (макроскопические) параметры системы, 
такие как плотность числа, гранулярная температура, средняя скорость потока и т.д. Самой важной особенностью данного решения
в том, что все подсистемы охлаждаются с одинаковой скоростью. Отсюда мы понимаем, что неодинаковость температур подсистем,
появляется в начальной стадии эволюции системы, затем можно предположить что они эволюционируют независимо друг от друга,
с различными начальными температурами, до тех пор пока системы не приходит к стационарному состоянию, при котором диссипация
компенсируется подкачкой энергии в систему, и все подсистемы приходят к своим собственным стационарным температурам.
Одинаковость скорости охлаждения выражается следующим образом:
\begin{equation}
  \frac{1}{T_i}\frac{dT_i}{dt}=\frac{1}{T_j}\frac{dT_j}{dt}=-\xi(t)\;.
\end{equation}

Используя нормальное решение системы, введем понятие \emph{среднего поля}. Это некая мнимая подсистема, с некоторой массой 
мнимых частиц $\bar{m}$, и некоторой мнимой температурой $\bar{T}$. Эти величины характеризуются полидисперсностью системы в целом
и имеют следующий вид:
\begin{equation}
  \begin{split}
    \bar{T} &= \int d\chi_iT_i\;,\\
    \bar{m} &= \int d\chi_im_i\;.
  \end{split}
\end{equation}
Теперь возьмем уравнение для охлаждения системы (\ref{eq:cooling_term}), и проинтегрируем по всему ансамблю:
\begin{equation}
  \begin{split}
    \frac{d}{dt}\int d\chi_iT_i &= -\int d\chi_iT_i\xi_i(t) = -\int d\chi_i\int d\chi_j\omega_{ij}(A_{ij}T_i-B_{ij}(T_j-T_i))=\\
    &=-\int d\chi_iT_i\int d\chi_j\omega_{ij}A_{ij}+\iint d\chi_id\chi_j\omega_{ij}B_{ij}(T_j-T_i)\;.
  \end{split}
\end{equation}
Так как параметр $B_{ij}$ отвечает за перенос энергии внутри системы, и не участвует в диссипации, то при интегрировании по
всей системе, последний интеграл исчезает, это также видно по тому факту, что выражение под интегралом является антисимметричным.
Таким образом, получаем:
\begin{equation}\label{eq:mean_field_temp_interm}
  \frac{d\bar{T}}{dt}=-\int d\chi_iT_i\int d\chi_j\omega_{ij}A_{ij}\;.
\end{equation}
С другой стороны, для нормального решения, мы также можем написать:
\begin{equation}
  \frac{dT_i}{dt}=-\xi(t)T_i\;,
\end{equation}
и интегрируя по всему ансамблю получаем:
\begin{equation}
  \int d\chi_i\frac{dT_i}{dt}=-\int d\chi_i\xi(t)T_i\;,
\end{equation}
или
\begin{equation}
  \frac{d\bar{T}}{dt}=-\xi(t)\bar{T}\;.
\end{equation}
Приравнивая к (\ref{eq:mean_field_temp_interm}), получаем:
\begin{equation}
  \xi(t) = \frac{\int d\chi_iT_i\int d\chi_j\omega_{ij}A_{ij}}{\int d\chi_iT_i}\;.
\end{equation}
Для дальнейшего продвижения, сделаем т.н. \emph{аппроксимацию по среднему полю}. Предположим, что всю систему 
можно заменить монодисперсной системой с макропараметрами, поведение которых такое же как и у изначальной системы.
В таком случае, напишем:
\begin{equation}
  \xi(t)\approx\frac{\bar{T}\cdot\int d\chi_i\omega_{ij}A_{ij}}{\bar{T}}=\bar{\omega}\bar{A}\;,
\end{equation}
где $\bar{\omega}$ -- средняя частота столкновений. Для колец Сатурна нам известно, что 
\begin{equation}
  \bar{\omega}\approx3\Omega\tau\;,
\end{equation}
где $\tau$ -- оптическая толщина кольца. Диссипативный параметр $\bar{A}$ имеет форму:
\begin{equation}
  \bar{A}\propto\frac{1-\eps^2}{2}\;,
\end{equation}
и в итоге можно написать:
\begin{equation}
  \xi(t)=\xi\approx\frac{3}{2}\left(1-\eps^2\right)\Omega\tau\;.
\end{equation}