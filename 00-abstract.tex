% Также можно использовать \Referat, как в оригинале
\begin{abstract}
    {\bf Компьютерное моделирование кинетического и гидродинамического приближения сложных статистических систем}

    Перечень  ключевых  слов:нейтронное  рассеяние, наносистемы  и  материалы, дифракция  нейтронов, 
    рентгеновская  дифракция, нейтронная  спектроскопия, камера  высокого  давления, импульсные  источники  нейтронов, 
    конструкционные материалы, высокотвердые сплавы, нанотрубки, каркасно-нанокластерные бориды, углеволокно, 
    высокотемпературные сверхпроводники, эластомеры. Объектами исследованияи разработкив данной работе являются 
    наносистемы и наноматериалы, твержые  сплавы, функциональные  материалы,  в  том  числе каркасно-нанокластерные  бориды,  
    композиты  из  углеродных  волокон,  карбид кремния, высокотемпературные сверхпроводники нового поколения 
    и родственные им   соединения,   моносилиципы   переходных   металлов,   сложные   оксиды, кобальтиты. 
    
    Целью данной работы является получение новых знаний и результатов в области структурных и динамических свойств 
    наносистем и наноматериалов, исследование наносистем и материалов методом рассеяния тепловых и эпитепловых нейтронов, 
    рентгеновской   дифракции,   обеспечение   научно-исследовательских   работ, проводимых  организациями  
    Российской  Федерации,  с  предоставлением  им возможности использования  методов  научных  исследований,  
    разработанных  или освоенных для уникальной установки –Нейтронного комплекса ИЯИ РАН. 
    
    Метод проведения работы: настоящая работа была выполнена при использовании нейтронных  методик  исследования  
    конденсированныхсред  в  сочетании  с комплементарными   рентгеновскими   методами. Использовались   нейтронная 
    дифракция, нейтронная    спектроскопия,    рентгеновская    дифракция, Мессбауэровская спектроскопия. 
    
    Результаты работы: На  Нейтронном  комплексе  ИЯИ  РАН, прочих  нейтронных  источниках,  
    на рентгеновских дифрактометрах в ИЯИ РАН, на Мёссбауэровском спектрометре в ИЯИ РАН были исследованы структурные 
    и динамические свойства материалов, в том числе наносистем, включающих в себя твердые сплавы с нановключениями, 
    каркасно-кластерные бориды  с  высокими  термоэлектрическими  свойствами, высокотемпературные  сверхпроводники  
    нового  поколения  и  родственные  им системы, сложные  оксиды  на  основе  переходных  металлов,  
    композитные материалы  на  основе  углеволокна  для  авиакосмических  приложений,  система углерод-кремний   
    с   высокими   механическими   качествами   и   химической стойкостью.Была  проведена  работа  по  дальнейшему  
    совершенствованию экспериментальной базы Нейтронного комплекса ИЯИ РАН, предназначенной для нейтронной спектроскопии 
    и нейтронной дифракции. В ходе работ по реализации задач этапа было привлечено в исследования по тематике 
    Госконтракта несколько студентов и аспирантов. 
    
    Основные   конструктивные,   технологические   и   
    технико-эксплуатационные характеристики: все  нейтронные  установки  Нейтронного  комплекса  ИЯИ  РАН 
    основаны на методике регистрации нейтронов по времени пролета. Особенностями источника  являются  относительно  
    жесткий  нейтронный  спектр  и  возможность вариации  длительности  импульса. 
    Важной  для  повышения  эффективности измерений  особенностью  рентгеновского  оборудования  ИЯИ РАН  
    является наличие позиционно-чувствительного детектора (imageplate). 
    
    Степень  внедрения: степень  внедрения  результатов  НИР  будет  выяснена  после завершения работ по Госконтракту. 
    
    Рекомендации по внедрению или итоги внедрения результатов НИР:рекомендации по  внедрению  результатов  
    НИР  будут  сделаны  после  завершения работ  по Госконтракту. Область применения:исследуемые наносистемы 
    и материалы будут применяться в энергетике, научном    приборостроении, химической    промышленности, 
    авиакосмической промышленности, атомной энергетике.
    
    Экономическая  эффективность  или  значимость  работы:оценка  экономической эффективности  и  значимости  
    работы  будет  сделана  после  завершения работ  по Госконтракту. 
    
    Прогнозные  предположения  о  развитии объекта  
    исследования: прогнозные предположения будут сделаны после завершения работ по Госконтракту. 
    
\end{abstract}

%%% Local Variables: 
%%% mode: latex
%%% TeX-master: "rpz"
%%% End: 
