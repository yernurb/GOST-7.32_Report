% Также можно использовать \Referat, как в оригинале
\begin{abstract}
    Отчет 34 с., 4 рис., 1 табл. 28 источн.

    ГРАНУЛЯРНЫЕ ГАЗЫ, КИНЕТИКА, ГИДРОДИНАМИКА, ПЛАНЕТАРНЫЕ КОЛЬЦА, ДИСКРЕТНЫЕ ЭЛЕМЕНТЫ

    Объектом исследования является гранулярный газ в виде двумерного диска с дифференциальным вращением, как планетарные кольца.

    Цель работы -- исследование влияния эффекта полидисперсности на эволюцию гранулярных температур подсистем газа.

    В процессе работы были созданы кинетические и гидродинамические модели для полидисперсного гранулярного газа, 
    также модель была расширена для анализа системы планетарных колец. Кроме теоретической модели, также построена
    компьютерная модель, для симуляции большого количества взаимодействующих частиц, с учетом их диссипативной природы
    и внешних условий.

    Предложенная модель полидисперсных гранулярных газов может помочь в более глубоком понимании процессов, происходящих
    в планетарных кольцах, таких как радиальная сегрегация по размерам частиц кольца и т.д.

    В дальнейшем предложенную модель можно расширить с учетом адгезивных взаимодействий. В этом случае можно будет уже
    рассуждать о самих агрегатов, и вопросе происхождения полидисперсности в целом.
    
\end{abstract}

%%% Local Variables: 
%%% mode: latex
%%% TeX-master: "rpz"
%%% End: 
