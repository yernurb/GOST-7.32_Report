\chapter{Диск с дифференциальным вращением}
\label{cha:impl}

 Мы вывели гидродинамические уравнения переноса и скорость охлаждения системы за счет столкновений.
 Выпишем эти уравнения заново:
 \begin{equation}
    \begin{split}
        \frac{\pd\rho}{\pd t}+\frac{\pd}{\pd r_{\alpha}}(\rho u_{\alpha}) &=0\;\\
        \frac{\pd(\rho u_{\alpha})}{\pd t} + \frac{\pd}{\pd r_{\beta}}(\rho u_{\alpha}u_{\beta}) 
        &= -\frac{\pd (nT)}{\pd r_{\alpha}}-\frac{\pd\pi_{\ab}}{\pd r_{\beta}}+\rho w_{\alpha}\;,\\
        \frac{\pd}{\pd t}\left(nT+\frac{\rho u_{\beta}u_{\beta}}{2}\right) + \frac{\pd}{\pd r_{\alpha}}u_{\alpha}
        \left(2nT+\frac{\rho u_{\beta}u_{\beta}}{2}\right) &= \rho w_{\beta}u_{\beta}-\frac{\pd}{\pd r_{\alpha}}(\pi_{\ab}u_{\beta})
        -\frac{\pd q_{\alpha}}{\pd r_{\alpha}}-T\xi\;.
    \end{split}
\end{equation}
Эти уравнения выписаны в общем виде для двумерной системы. Рассматриваемая нами система является плоским диском
с дифференциальным вращением и с азимутальной симметрией. Поэтому займемся упрощением выведенных гидродинамических уравнений.

\section{Азимутальная симметрия}
Сделаем переход на полярные координаты $x,\,y\to r,\,\theta$, где $r$ -- расстояние до центра планеты, $\theta$ -- 
азимутальная позиция, тогда дифференциальное вращение означает что угловая скорость вращения зависит от $r$, т.е.
в нашем случаем кеплеровской, а азимутальная симметрия означает что макропараметры системы не зависят от $\theta$:
\begin{equation}
    f=f(t,\,m,\,r,\,\dot{r},\,\dot\theta)\;.
\end{equation}
Гравитационный потенциал имеет следующий вид:
\begin{equation}
    U(r) = -\frac{GM_P}{r}\;,
\end{equation}
где $G$ -- гравитационная постоянная,  $M_P$ -- масса планеты. Ускорение частицы в этом поле равна:
\begin{equation}
     \bw = -\frac{\pd U(r)}{\pd r}\frac{\br}{r}=-\frac{GM_P}{r^2}\be_r = 
     \left(\begin{array}{c}
         -GM_P/r^2 \\ 0
     \end{array}\right)\;,
\end{equation}
где $\be_r$ -- единичный вектор в радиальном направлении. Средняя скорость потока на кеплеровской орбите равна:
\begin{equation}
    \bu = \Omega r\cdot\be_{\theta}=\left(\begin{array}{c}
        0 \\ \Omega r
    \end{array}\right)\;,
\end{equation}
где $\be_{\theta}$ -- единичный вектор в азимутальном направлении. Также, векторный оператор набла в полярных координатах
имеет следующий вид:
\begin{equation}
    \nabla = \frac{\pd}{\pd r_{\alpha}} = \frac{\pd}{\pd r}\be_r+\frac{1}{r}\frac{\pd}{\pd\theta}\be_{\theta}=
    \left(\begin{array}{c}
        \frac{\pd}{\pd r} \\ \frac{1}{r}\frac{\pd}{\pd\theta}
    \end{array}\right)\;.
\end{equation}
Таким образом, уравнение непрерывности принимает вид:
\begin{equation}
    \frac{\pd\rho}{\pd t}+\frac{1}{r}\frac{\pd(\rho\Omega r)}{\pd\theta} = 0\;,
\end{equation}
и далее получаем совершенно естественный результат для нашей системы:
\begin{equation}
    \frac{\pd\rho}{\pd t} = 0\;.
\end{equation}

Теперь рассмотрим уравнение переноса импульса. Для этого сначала необходимо вывести выражения для нескольких тензоров.
Первый тензор получается как внешнее произведение двух векторов:
\begin{equation}
\rho u_iu_j = \rho\left(\begin{array}{c} 0 \\ \Omega r\end{array}\right)\circ\left(\begin{array}{cc} 0 & \Omega r\end{array}\right)=
    \left(\begin{array}{cc}
    0 & 0 \\
    0 & \rho\Omega^2r^2
    \end{array}\right)\;,
\end{equation}
и далее получаем:
\begin{equation}
    \frac{\pd}{\pd r_{\beta}}(\rho u_{\alpha}u_{\beta})=\left(\begin{array}{c}
        \frac{\pd}{\pd r} \\ \frac{1}{r}\frac{\pd}{\pd\theta}
    \end{array}\right)
    \left(\begin{array}{cc}
        0 & 0 \\
        0 & \rho\Omega^2r^2
    \end{array}\right)=0\;.
\end{equation}
Второй, это тензор вязких напряжений (\ref{eq:viscosity_coeff}). В нашем случае он имеет вид:
\begin{equation}
    \pi_{\ab} = -\nu\left(\begin{array}{cc}
        0 & \frac{\pd(\Omega r)}{\pd r} \\
        \frac{\pd(\Omega r)}{\pd r} & 0
    \end{array}\right)\;,
\end{equation}
и так как:
\begin{equation}
    \frac{\pd(\Omega r)}{\pd r} = -\frac{\Omega}{2}\;,
\end{equation}
получаем:
\begin{equation}
    \pi_{\ab} = \frac{1}{2}\left(\begin{array}{cc}
        0 & \nu\Omega \\
        \nu\Omega & 0
    \end{array}\right)\;.
\end{equation}
Его пространственные производные имеют вид:
\begin{equation}
    \frac{\pd\pi_{\ab}}{\pd r_{\beta}}=\frac{1}{2}\left(\begin{array}{c}
        \frac{\pd}{\pd r} \\ \frac{1}{r}\frac{\pd}{\pd\theta}
    \end{array}\right)
    \left(\begin{array}{cc}
        0 & \nu\Omega \\
        \nu\Omega & 0
    \end{array}\right)\;,
\end{equation}
и так как:
\begin{equation}
    \frac{\pd\Omega}{\pd r}=-\frac{3}{2}\frac{\Omega}{r}\;,
\end{equation}
в итоге получаем:
\begin{equation}
    \frac{\pd\pi_{\ab}}{\pd r_{\beta}}=-\frac{3}{4}\frac{\nu\Omega}{r}\be_{\theta}=-\left(\begin{array}{c}
        0 \\ \frac{3}{4}\frac{\nu\Omega}{r}
    \end{array}\right)\;.
\end{equation}
В конечном итоге, уравнение переноса импульса принимает вид:
\begin{equation}
    \rho\frac{\pd\bu}{\pd t} = -\nabla(nT)+\frac{3}{4}\frac{\nu}{r^2}\bu-\rho\nabla U(r)\;,
\end{equation} 
которое также тривиальным образом обнуляется.
Теперь рассмотрим уравнение переноса энергии. Необходимые в этом случае тензорные операции выглядят следующим образом:
\begin{equation}
    \pi_{\ab}u_{\beta} = \frac{1}{2}\left(\begin{array}{cc}
        0 & \nu\Omega \\
        \nu\Omega & 0
    \end{array}\right)\left(\begin{array}{c}
        0 \\ \Omega r        
    \end{array}\right) = \frac{1}{2}\left(\begin{array}{cc}
        \nu\Omega^2 r & 0        
    \end{array}\right)\;,
\end{equation}
и соответственно:
\begin{equation}
    \frac{\pd(\pi_{\ab}u_{\beta})}{\pd r_{\alpha}} = -\nu\Omega^2\;.
\end{equation}
Поток тепла записывается:
\begin{equation}
    \frac{\pd q_{\alpha}}{\pd r_{\alpha}} = -\lambda\Delta T\;.
\end{equation}
Так как оператор Лапласа в полярных координатах выглядят следующим образом:
\begin{equation}
    \Delta T = \frac{\pd^2 T}{\pd r^2}+\frac{1}{r}\frac{\pd T}{\pd r}+\frac{1}{r^2}\frac{\pd^2 T}{\pd\theta^2}\;.
\end{equation}
то в конечном итоге получаем уравнение переноса энергии в системе:
\begin{equation}\label{eq:temperature_evolution}
    n\frac{\pd T}{\pd t} = \nu\Omega^2 + \lambda\frac{\pd^2T}{\pd r^2} + \frac{\lambda}{r}\frac{\pd T}{\pd r} - T\xi\;.
\end{equation}
Анализируя полученное уравнение эволюции температуры в системе, видим что она зависит от подкачки
гравитационной энергии $\nu\Omega^2$, диссипации $-\xi$, и также переноса тепла в радиальном направлении $\lambda\Delta T$.

\section{Стационарное решение}
Произведем анализ стационарного решения уравнения (\ref{eq:temperature_evolution}). Если пренебречь радиальной зависимостью
температуры, то получаем:
\begin{equation}
    \left(k_1\nu_{l}+k_2\nu_{nl}\right)\Omega^2=k_3\left(1-\eps^2\right)\Omega\tau\cdot\bar{T}\;,
\end{equation}
где $k_1,\,k_2,\,k_3$ -- некоторые константы. Коэффициент вязкости мы разделили на две части, т.н. локальную и нелокальную:
\begin{equation}
    \begin{split}
        \nu_{l}&\approx\frac{\bar{T}}{\Omega\bar{m}}\frac{\tau}{1+\tau^2}\;,\\
        \nu_{nl}&\approx\Omega\bar{\sigma}^2\tau\;,
    \end{split}
\end{equation}
где
\begin{equation}
  \bar{\sigma}=\int d\chi_i\sigma_i\;,
\end{equation}
среднее сечение. Теперь можно написать стационарное значение температуры среднего поля:
\begin{equation}
    \bar{T} = \frac{k_2\cdot\bar{m}\Omega^2\bar{\sigma}^2\left(1+\tau^2\right)}{k_3\cdot\bar{m}\left(1-\eps^2\right)\left(1+\tau^2\right)-k_1}\;.
\end{equation}

%В данном разделе описано изготовление и требование всячины. Кстати,
%в Latex нужно эскейпить подчёркивание (писать <<\verb|some\_function|>> для \Code{some\_function}).

%\ifPDFTeX
%Для вставки кода есть пакет \Code{listings}. К сожалению, пакет \Code{listings} всё ещё
%работает криво при появлении в листинге русских букв и кодировке исходников utf-8.
%В данном примере он (увы) на лету конвертируется в koi-8 в ходе сборки pdf.

%Есть альтернатива \Code{listingsutf8}, однако она работает лишь с
%\Code{\textbackslash{}lstinputlisting}, но не с окружением \Code{\textbackslash{}lstlisting}

%Вот так можно вставлять псевдокод (питоноподобный язык определен в \Code{listings.inc.tex}):

%\begin{lstlisting}[style=pseudocode,caption={Алгоритм оценки дипломных работ}]
%def EvaluateDiplomas():
%    for each student in Masters:
%        student.Mark := 5
%    for each student in Engineers:
%        if Good(student):
%            student.Mark := 5
%        else:
%            student.Mark := 4
%\end{lstlisting}

%Еще в шаблоне определен псевдоязык для BNF:

%\begin{lstlisting}[style=grammar,basicstyle=\small,caption={Грамматика}]
%  ifstmt -> "if" "(" expression ")" stmt |
%            "if" "(" expression ")" stmt1 "else" stmt2
%  number -> digit digit*
%\end{lstlisting}

%В листинге~\ref{lst:sample01} работают русские буквы. Сильная магия. Однако, работает
%только во включаемых файлах, прямо в \TeX{} нельзя.

% Обратите внимание, что включается не ../src/..., а inc/src/...
% В Makefile есть соответствующее правило для inc/src/*,
% которое копирует исходные файлы из ../src и конвертирует из UTF-8 в KOI8-R.
% Кстати, поэтому использовать можно только русские буквы и ASCII,
% весь остальной UTF-8 вроде CJK и египетских иероглифов -- нельзя.

%\lstinputlisting[language=C,caption=Пример (\Code{test.c}),label=lst:sample01]{listings/test.c}

%\else

%Для вставки кода есть пакет \texttt{minted}. Он хорош всем кроме: необходимости Python (есть во всех нормальных (нет, Windows, я не про тебя) ОС) и Pygments и того, что нормально работает лишь в \XeLaTeX.

%Можно пользоваться расширенным BFN:

%\begin{listing}[H]
%\begin{ebnfcode}
% letter = "A" | "B" | "C" | "D" | "E" | "F" | "G"
%       | "H" | "I" | "J" | "K" | "L" | "M" | "N"
%       | "O" | "P" | "Q" | "R" | "S" | "T" | "U"
%       | "V" | "W" | "X" | "Y" | "Z" ;
%digit = "0" | "1" | "2" | "3" | "4" | "5" | "6" | "7" | "8" | "9" ;
%symbol = "[" | "]" | "{" | "}" | "(" | ")" | "<" | ">"
%       | "'" | '"' | "=" | "|" | "." | "," | ";" ;
%character = letter | digit | symbol | "_" ;
 
%identifier = letter , { letter | digit | "_" } ;
%terminal = "'" , character , { character } , "'" 
%         | '"' , character , { character } , '"' ;
 
%lhs = identifier ;
%rhs = identifier
%     | terminal
%     | "[" , rhs , "]"
%     | "{" , rhs , "}"
%     | "(" , rhs , ")"
%     | rhs , "|" , rhs
%     | rhs , "," , rhs ;
 
%rule = lhs , "=" , rhs , ";" ;
%grammar = { rule } ;
%\end{ebnfcode}
%\caption{EBNF определённый через EBNF}
%\label{lst:ebnf}
%\end{listing}

%А вот в листинге \ref{lst:c} на языке C работают русские комменты. Спасибо Pygments и Minted за это.

%\begin{listing}[H]
%\cfile{inc/src/test.c}
%\caption{Пример — test.c} 
%\end{listing}
%\label{lst:c}

%\fi

% Для вставки реального кода лучше использовать \texttt{\textbackslash lstinputlisting} (который понимает
% UTF8) и стили \Code{realcode} либо \Code{simplecode} (в зависимости от размера куска).




%Можно также использовать окружение \Code{verbatim}, если \Code{listings} чем-то не
%устраивает. Только следует помнить, что табы в нём <<съедаются>>. Существует так же команда \Code{\textbackslash{}verbatiminput} для вставки файла.

%\begin{verbatim}
%a_b = a + b; // русский комментарий
%if (a_b > 0)
%    a_b = 0;
%\end{verbatim}

%%% Local Variables:
%%% mode: latex
%%% TeX-master: "rpz"
%%% End:
